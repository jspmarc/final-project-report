%--------------------------------------------------------------------%
%
% Berkas utama templat LaTeX.
%
% author Petra Barus, Peb Ruswono Aryan, Faris Rizki Ekananda
%
%--------------------------------------------------------------------%
%
% Berkas ini berisi struktur utama dokumen LaTeX yang akan dibuat.
%
%--------------------------------------------------------------------%

\documentclass[bahasa, 12pt, a4paper, onecolumn, oneside, final]{report}

%-------------------------------------------------------------------%
%
% Konfigurasi dokumen LaTeX untuk laporan tesis IF ITB
%
% @author Petra Novandi
%
%-------------------------------------------------------------------%
%
% Berkas asli berasal dari Steven Lolong
%
%-------------------------------------------------------------------%

% Ukuran kertas
\special{papersize=210mm,297mm}

% Setting margin
\usepackage[top=3cm,bottom=3cm,left=4cm,right=3cm]{geometry}

\usepackage{mathptmx}

% Judul bahasa Indonesia
\usepackage[bahasa]{babel}

% Format citation
\usepackage[style=apa,backend=biber]{biblatex}

\usepackage[utf8]{inputenc}
\usepackage{graphicx}
\usepackage{titling}
\usepackage{blindtext}
\usepackage{sectsty}
\usepackage{chngcntr}
\usepackage{etoolbox}
\usepackage{hyperref}       % Package untuk link di daftar isi. Ubah jadi \usepackage[hidelinks]{hyperref} apabila ingin menghilangkan kotak merah disekitar link
\usepackage{titlesec}       % Package Format judul
\usepackage{titletoc}       % Package Format judul di toc
\usepackage{tocbibind}      % Package untuk masukkan toc, lot, lof ke Daftar Isi
\usepackage{scrwfile}       % Package untuk membuat Daftar Lampiran dari toc
\usepackage{parskip}
\usepackage{afterpage}
\usepackage{relsize}
\usepackage{listings}       % Package lstlistings and stuff
\usepackage{xcolor}

\graphicspath{{resources/}}   % letak direktori penyimpanan gambar

% Setting daftar lampiran
\newcommand*{\lopname}{DAFTAR LAMPIRAN}
\TOCclone[\lopname]{toc}{atoc}
\addtocontents{atoc}{\protect\value{tocdepth}=-1}
\newcommand\listofappendices{
  \cleardoublepage
  \phantomsection
  \listofatoc
  \addcontentsline{toc}{chapter}{\lopname}
}

\newcommand*\savedtocdepth{}
\AtBeginDocument{%
  \edef\savedtocdepth{\the\value{tocdepth}}%
}

\let\originalappendix\appendix
\renewcommand\appendix{%
  \originalappendix
  \cleardoublepage
  \addtocontents{toc}{\protect\value{tocdepth}=-1}%
  \addtocontents{atoc}{\protect\value{tocdepth}=\savedtocdepth}%

  \titlecontents{chapter}
    [0pt]
    {\bfseries}
    {Lampiran \thecontentslabel.\quad}
    {}
    {\hfill\contentspage}

  \titleformat{\chapter}[block]
    {\bfseries}
    {\chaptertitlename\ \thechapter.\quad}{0pt}
    {\bfseries}
}

% Define colors
\colorlet{punct}{red!60!black}
\definecolor{delim}{RGB}{20,105,176}
\colorlet{numb}{magenta!60!black}
\definecolor{eclipseStrings}{RGB}{42,0.0,255}
\definecolor{eclipseKeywords}{RGB}{127,0,85}

% Hilangkan titik pada toc
\makeatletter
\renewcommand{\@dotsep}{10000} 
\makeatother

% Setel title pada chapter-chapter di toc, lof, lot
\titlecontents{chapter}
  [0pt]
  {\bfseries}
  {\MakeUppercase{Bab} \thecontentslabel\quad\uppercase}
  {}
  {\hfill\contentspage}
\titlecontents{figure}
  [0pt]
  {}
  {Gambar \thecontentslabel.\quad}
  {}
  {\hfill\contentspage}
\titlecontents{table}
  [0pt]
  {}
  {Tabel \thecontentslabel.\quad}
  {}
  {\hfill\contentspage}

% Masukin Daftar Pustaka ke toc
\let\originalprintbibliography\printbibliography
\renewcommand\printbibliography{%
  \phantomsection
  \cleardoublepage
  \originalprintbibliography
  \addcontentsline{toc}{chapter}{\bibname}
}

% Line satu setengah spasi
\renewcommand{\baselinestretch}{1.5}

% Setting judul
\chapterfont{\centering \Large}
\titleformat{\chapter}[display]
  {\Large\centering\bfseries}
  {\chaptertitlename\ \thechapter}{0pt}
    {\Large\bfseries\uppercase}

% Setting nomor pada subbsubsubbab
\setcounter{secnumdepth}{3}

\makeatletter

\makeatother

% Counter untuk figure dan table.
\counterwithin{figure}{section}
\counterwithin{table}{section}

% Define blank page
\newcommand*{\blankpage}{\afterpage{\null\newpage}}

% Set code snippet settings
\lstset{
  aboveskip=3mm,
  basicstyle={\small\ttfamily},
  belowskip=3mm,
  captionpos=b,
  columns=flexible,
  frame=tb,
  framexleftmargin=20pt,
  showstringspaces=false,
  numbers=left,
  numbersep=5pt,
  numberstyle=\tiny\color{gray},
  keywordstyle=\color{blue},
  commentstyle=\color{dkgreen},
  stringstyle=\color{eclipseStrings},
  breaklines=true,
  breakatwhitespace=true,
  tabsize=3,
  xleftmargin=20pt
}

\lstdefinelanguage{JSON}{
    basicstyle=\normalfont\ttfamily,
    commentstyle=\color{eclipseStrings}, % style of value
    stringstyle=\color{eclipseKeywords}, % style of key
    numbers=left,
    numberstyle=\scriptsize,
    stepnumber=1,
    numbersep=8pt,
    showstringspaces=false,
    breaklines=true,
    frame=lines,
    string=[s]{"}{"},
    comment=[l]{:\ "},
    morecomment=[l]{:"},
    literate=
        *{0}{{{\color{numb}0}}}{1}
         {1}{{{\color{numb}1}}}{1}
         {2}{{{\color{numb}2}}}{1}
         {3}{{{\color{numb}3}}}{1}
         {4}{{{\color{numb}4}}}{1}
         {5}{{{\color{numb}5}}}{1}
         {6}{{{\color{numb}6}}}{1}
         {7}{{{\color{numb}7}}}{1}
         {8}{{{\color{numb}8}}}{1}
         {9}{{{\color{numb}9}}}{1}
}

% Setting daftar kode (?) TODO: konfirmasi boleh ada snippet kode atau ngga
\renewcommand{\lstlistingname}{Kode}
\renewcommand{\lstlistlistingname}{\uppercase{Daftar Kode}}


\makeatletter

\makeatother

\addbibresource{references.bib}

\begin{document}

%Basic configuration
\title{Pengembangan Simulasi \textit{Hardware-in-the-loop} Kendaraan Otonom
	Menggunakan CARLA}
\date{23 Juni 2023}
\author{
	Josep Marcello \\
	NIM : 13519164
}

\pagenumbering{roman}
\setcounter{page}{1}

\input{chapters/frontmatter/cover}
\input{chapters/approval/approval}
\clearpage
\pagestyle{empty}

\begin{center}
	\smallskip

	\Large \bfseries \MakeUppercase{
		Lembar Identitas \\
		Tugas Akhir Capstone
	}
	\vspace{0.5cm}

	\raggedright
	\begin{table}[h!]
		\large \bfseries
		\begin{tabular}{p{0.3\textwidth} p{0.63\textwidth}}
			Judul Proyek TA : & \capstoneTitle
		\end{tabular}
	\end{table}

	\normalsize \normalfont

	Anggota tim dan pembagian peran:

	\begin{table}[h!]
		\begin{tabular}{|p{0.05\textwidth} | p{0.13\textwidth} | p{0.19\textwidth} | p{0.50\textwidth}|}
			\hline
			\textbf{No.} & \textbf{NIM} & \textbf{Nama}         & \textbf{Peran}                                                                                                 \\
			\hline
			1.           & 13519116     & Jeane Mikha Erwansyah & Pengimplementasian Objek Lokal dan Lingkungan Indonesia untuk Simulasi Trem Otonom Menggunakan Simulator CARLA \\
			\hline
			2.           & 13519164     & Josep Marcello        & Pembuatan Jalur Komunikasi Antara Simulator CARLA dengan Server NVIDIA Pegasus                                 \\
			\hline
			3.           & 13519188     & Jeremia Axel Bachtera & Pembangunan Skenario Pengujian Simulasi Trem Otonom di Indonesia                                               \\
			\hline
		\end{tabular}
	\end{table}

	\vfill
	\pembimbingTtd

\end{center}
\clearpage


\pagestyle{plain}

\chapter*{ABSTRAK}
\addcontentsline{toc}{chapter}{\MakeUppercase{Abstrak}}

%taruh abstrak bahasa indonesia di sini
\begin{center}
	\bfseries \MakeUppercase{\thetitle}

	\normalfont\normalsize
	Oleh:

	\theauthor
\end{center}

\begin{singlespace}
	Dengan perkembangan pesat kota-kota di Indonesia dan semakin besarnya
	kebutuhan untuk mengurangi masalah lalu lintas dan masalah lingkungan,
	dibutuhkan transportasi umum yang efisien dalam membawa penumpang, aman, dan
	murah. Salah satu transportasi ini adalah trem otonom. Trem otonom tidak
	membutuhkan masinis dalam operasinya. Artinya, biaya operasional trem dapat
	lebih menjadi murah dan keamanan dapat lebih dijamin.

	Akan tetapi, pengembangan trem otonom ini membutuhkan biaya yang banyak dan
	waktu yang lama jika dilakukan secara langsung. Oleh karena itu akan
	digunakan sebuah sistem simulasi yang dapat menguji perangkat keras dan
	perangkat lunak yang digunakan pada trem otonom nanti. Skema simulasi
	tersebut dapat disebut juga dengan \textit{hardware-in-the-loop-simulation}
	(HILS). Pada sistem HILS akan dimanfaatkan simulator CARLA untuk menjalankan
	dunia virtual dan mendapatkan data dari sensor virtual.

	Pada proyek pengembangan trem otonom ini, sudah ada sistem HILS. Akan
	tetapi, belum ideal karena kinerja yang buruk dan tidak dapat menggunakan
	data sensor. Kinerja buruk ini disebabkan banyaknya operasi I/O dan
	arsitektur sistem yang lebih kompleks dari seharusnya. Oleh karena itu,
	akan dibuat sebuah sistem HILS baru untuk menyelesaikan kedua masalah
	tersebut.

	Sistem HILS baru yang diimplementasikan berhasil memiliki latensi setidaknya
	2,5x lebih cepat dari sistem HILS lama. Selain itu, simulasi berhasil
	menggunakan data sensor virtual dan simulator CARLA berhasil berjalan dengan
	lebih dari 2 FPS.

	\textbf{\textit{Kata kunci: HILS, sistem simulasi, kendaraan otonom}}
\end{singlespace}
\clearpage
% \clearpage
\chapter*{Abstract}
\addcontentsline{toc}{chapter}{\MakeUppercase{Abstract}}

%put your abstract here
\begin{center}
	\bfseries \MakeUppercase{\thetitleEn}

	\normalfont\normalsize
	By:

	\theauthor
\end{center}

\begin{singlespace}
	With the rapid growth of cities in Indonesia and the need to reduce impacts
	of growth to traffic and the environment, a public transport that can
	efficiently carry passengers, safe, and inexpensive.  One of these public
	transport is autonomous tram. The autonomous tram doesn't need a driver to
	operate it. This results in cheaper operational cost and safety can be
	better guaranteed.

	However, the development of autonomous tram is expensive and needs a lot of
	time if the testings were done in a real-world environment. Hence, a
	simulation system that an test the hardware and software of the autonomous
	tram is used for testing and development. A simulation that tests the
	hardware and software is also known as hardware-in-the-loop-simulation
	(HILS). In the HILS system, the CARLA simulator is used to run the virtual
	world and to get data from virtual sensors.

	In the autonomous tram development project, a HILS system alrady exists.
	However, the system is not yet ideal since its performance is bad and sensor
	data can't be used in the simulation system. The bad performance is caused
	by heaps of I/O operations for communication and the system's architecture
	which is more complicated than it needs to be. Therfore, a new communication
	mechanism is created for the HILS system to solve the bad performance and
	inability to use sensor data.

	The new communication mechanism is at least 2.5x faster when compared to the
	old HILS system's communication mechanism. Moreover, the simulation system
	can use sensor data  while keeping CARLA running with at least 2 FPS.

	\textbf{\textit{Keywords: HILS communication, CARLA for HILS, autonomous
			vehicle simulation}}
\end{singlespace}

\clearpage

\input{chapters/frontmatter/statement}
\chapter*{Kata Pengantar}
\addcontentsline{toc}{chapter}{Kata Pengantar}

Puji dan Syukur Penulis panjatkan ke Tuhan Yang Maha Esa karena atas kasih dan
berkat-Nya Penulis dapat menyelesaikan \textit{draft} laporan tugas akhir yang
berjudul ``\thetitle.'' Kasih dan berkat Tuhan juga muncul dari berbagai pihak
yang membantu dan mendukung penulis dalam mengerjakan dan menyelesaikan tugas
akhir ini. Penulis sangat berterima kasih kepada semua pihak yang sudah membantu
penulis dalam mengerjakan tugas akhir ini, terutama

\begin{enumerate}
	\item Pak \pembimbingSatu\ dan Pak \pembimbingDua\ selaku dosen pembimbing
	      tugas akhir penulis. Penulis berterima kasih atas semua bimbingan, masukan,
	      arahan sehingga tugas akhir ini dapat dikerjakan sebaik mungkin.
	\item Pak Handoko Supeno S.T., M.T. selaku asisten Penulis selama
	      mengerjakan tugas akhir dan juga selaku pemimpin tim simulasi di proyek
	      \textit{autonomous tram}. Penulis berterma kasih atas semua bantuan dan
	      arahan yang diberikan selama pengerjaan tugas akhir ini.
	\item Ayah, Ibu, dan adik-adik penulils yang sudah selalu memberikan doa dan
	      dukungan pada penulis. Serta selalu semangat menanyakan kabar Penulis.
	\item Tim simulasi \textit{autonomous tram}, yaitu rekan penulis dalam tugas
	      akhir \textit{capstone} ini: Jeremia Axel dan Jeane Mikha karena sudah
	      berjuang bersama untuk menamatkan tugas akhir ini.
	\item Seluruh dosen Teknik Informatika ITB serta dosen-dosen TPB Penulis
	      karena sudah mau menuangkan ilmu-ilmu berharganya kepada Penulis.
	\item Teman-teman dekat Penulis selama di Informatika ITB, terutama Felica,
	      Suggoi, Nathan, Alex, Ko Aloy, Ariya, Govind, dan masih banyak lagi, karena
	      sudah menemani, menyemangati, dan memotivasi Penulis selama setidaknya 3
	      tahun terakhir.
	\item Teman-teman TPB Penulis, terutama Reinaldo, Kelvin, dan Rehagana,
	      karena sudah menjadi teman-teman pertama Penulis di Bandung dan di ITB.
	\item Teman-teman doa Penulis: Helena, CB, dan Nana karena sudah mendoakan
	      Penulis dan karena sudah mengingatkan Penulis kepada Tuhan, kekuatan
	      doa, dan sikap berpasrah.
\end{enumerate}


\titleformat*{\section}{\centering\bfseries\Large\MakeUpperCase}
\titlespacing*{\chapter}{0pt}{0pt}{3pc}

% Setting judul toc, lot, lof, bib
\renewcommand{\contentsname}{DAFTAR ISI}
\renewcommand{\listfigurename}{DAFTAR GAMBAR}
\renewcommand{\listtablename}{DAFTAR TABEL}
\renewcommand{\bibname}{DAFTAR PUSTAKA}

\tableofcontents
\listofappendices
\listoffigures
\listoftables
\chapter*{Daftar Istilah}
\addcontentsline{toc}{chapter}{\MakeUppercase{Daftar Istilah}}

\begin{table*}[!ht]
	\centering
	\begin{tabular}{|l|l|l|}
		\hline
		\textbf{Istilah}           & \textbf{Penjelasan}                          & \textbf{Nomor}   \\
		                           &                                              & \textbf{Halaman} \\
		\hline
		HILS                       & \textit{hardware-in-the-loop-simulation}     & 2                \\
		\hline
		Komputer AGX               & NVIDIA Pegasus, komputer yang                & 2                \\
		                           & akan digunakan pada trem otonom.             &                  \\
		                           & Menjalankan program                          &                  \\
		                           & utama GRS.                                   &                  \\
		\hline
		Komputer RKB               & Komputer untuk pengujian program             & 2                \\
		                           & yang akan digunakan pada AGX.                &                  \\
		\hline
		Komputer SILS              & Komputer yang menjalankan CARLA              & 2                \\
		                           & dan program ScenarioRunner. Tidak            &                  \\
		                           & ada hubungan langsung dengan                 &                  \\
		                           & istilah SILS.                                &                  \\
		\hline
		NPC (\textit{non-playable} & Istilah untuk karakter pada gim yang         & 8                \\
		\textit{character})        & tidak dapat dimainkan.                       &                  \\
		\hline
		Program GRS                & Program utama berisi AI                      & 3                \\
		                           & dan algoritma \textit{decision making} trem. &                  \\
		\hline
		Program                    & Program utama dan dari CARLA                 & 3                \\
		ScenarioRunner             & untuk menjalankan skenario simulasi.         &                  \\
		\hline
		\textit{Round-trip-time}   & Waktu dari mengirim paket sampai             & 32               \\
		(RTT)                      & diterima balasan paket                       &                  \\
		\hline
		SILS                       & \textit{software-in-the-loop-simulation}     & 12               \\
		\hline
	\end{tabular}
\end{table*}

\newpage

\titleformat*{\section}{\bfseries\large}
\pagenumbering{arabic}

%----------------------------------------------------------------%
% Konfigurasi Bab
%----------------------------------------------------------------%
\setcounter{page}{1}
\renewcommand{\chaptername}{BAB}
\renewcommand{\thechapter}{\Roman{chapter}}
%----------------------------------------------------------------%

%----------------------------------------------------------------%
% Dafter Bab
% Untuk menambahkan daftar bab, buat berkas bab misalnya `chapter-6` di direktori `chapters`, dan masukkan ke sini.
%----------------------------------------------------------------%
\chapter{Pendahuluan}

\section{Latar Belakang}

Pada tahun 2015, Perserikatan Bangsa-bangsa (PBB) mengeluarkan 17 SDG
(\textit{sustainable development goal}). Salah satu target yang ingin dicapai
SGD nomor 11 adalah menyediakan sistem transportasi yang aman, murah, mudah
diakses, dan berkelanjutan (\textit{sustainable}) untuk khalayak ramai. Untuk
mendukung tujuan tersebut, dikembangkan trem untuk digunakan sebagai
transportasi publik di Indonesia. Trem dipilih karena murah dan menggunakan
listrik untuk bahan bakarnya sehingga dapat menggunakan energi yang ramah
lingkungan.

Trem diharapkan dapat menjadi alternatif bagi transportasi pribadi. Transportasi
pribadi semakin lama semakin tidak cocok untuk pembangunan berkelanjutan. Selain
alternatif, trem yang dikembangkan diharapkan juga dapat mengurangi masalah yang
ditimbulkan oleh kendaraan pribadi. Masalah-masalah tersebut termasuk, tapi
tidak terbatas pada, kemacetan dan emisi karbon dioksida.

Trem akan memanfaatkan teknologi inteligensi buatan dan pembelajaran mesin yang
memungkinkan kendaraan untuk beroperasi tanpa masinis secara otonom
(\textit{autopilot}). Dengan menggunakan teknologi ini, diharapkan trem otonom
dapat meminimalisasi kecelakaan karena galat yang disebabkan manusia sehingga
bisa menjadi pilihan transportasi yang aman.

Untuk mengembangkan teknologi pembelajaran mesin dan algoritma kendali
o\-to\-nom yang digunakan, perlu dikumpulkan data. Pengumpulan data ini akan
sangat mahal, butuh waktu yang lama, dan berbahaya bila dilakukan langsung di
lapangan. Oleh karena itu, data yang dikumpulkan akan dilakukan melalui
simulasi. Tugas akhir ini akan membahas pengembangan simulasi yang digunakan
untuk pengembangan trem otonom. Algoritma dan pembelajaran mesin juga akan
diujikan pada simulasi yang dibuat.

Sistem simulasi yang dibuat akan membutuhkan dua komputer:
\begin{enumerate}
	\item komputer untuk menjalankan simulasi (disebut server CARLA), dan
	\item komputer yang akan digunakan pada trem dan mengandung algoritma
	      kendali serta pembelajaran mesin (disebut server NVIDIA Pegasus).
\end{enumerate}
Sistem simulasi yang dibuat, selain harus dapat mereplikasi keadaan dunia nyata,
harus cukup cepat dalam memproses dan mengirimkan data antarkomputer. Hal metode
pengiriman data antara kedua komputer tersebut akan dibahas pada tugas akhir
ini.

Server NVIDIA Pegasus yang digunakan pada sistem simulasi ini juga akan
digunakan pada produk akhir trem otonom. Artinya, simulasi yang dilakukan
berbentuk HILS (\textit{hardware-in-the-loop simulation}).

\section{Rumusan Masalah}

Pada keadaan saat ini, server CARLA dan server NVIDIA Pegasus dihubungkan
menggunakan layanan web (\textit{web service}). Layanan web yang ada menggunakan
protokol HTTP untuk komunikasinya. Penggunaan HTTP untuk komunikasi pada layanan
web menyebabkan latensi yang besar ketika melakukan pertukuran data
antarserver. Karena adanya latensi besar ini, sistem simulasi menjadi tidak reliabel.

Berdasarkan latar belakang serta keadaan saat ini, definisi rumusan masalah
untuk tugas akhir ini adalah sebagai berikut.
\begin{enumerate}
	\item Bagaimana cara mengimplementasikan jalur komunikasi yang reliabel
	      untuk kebutuhan simulasi kendaraan otonom?
	\item Protokol atau kakas komunikasi apa yang paling tepat digunakan untuk
	      sistem simulasi kendaraan otonom?
\end{enumerate}

\section{Tujuan}

Tugas akhir ini memiliki 2 tujuan utama, yaitu
\begin{enumerate}
	\item mengimplementasikan jalur komunikasi yang reliabel untuk sistem
	      simulasi kendaraan otonom, dan
	\item menentukan protokol atau kakas komunikasi yang paling tepat untuk
	      sistem simulasi kendaraan otonom.
\end{enumerate}

\section{Batasan Masalah}

Batasan pada pengerjaan tugas akhr adalah

\begin{enumerate}
	\item pengujian akan dilakukan pada komputer yang terdapat di gedung Lab
	      Tek\-no\-lo\-gi VIII ITB;
	\item hanya ada 3 protokol atau kakas komunikasi yang diuji: ROS, ZeroMQ,
	      dan Apache Thrift; dan
	\item jalur komunikasi hanya akan dinilai berdasarkan latensi.
\end{enumerate}

\section{Metodologi}

% Tuliskan semua tahapan yang akan dilalui selama pelaksanaan tugas akhir.
% Tahapan ini spesifik untuk menyelesaikan persoalan tugas akhir. Tahapan studi
% literatur tidak perlu dituliskan karena ini adalah pekerjaan yang harus Anda
% lakukan selama proses pelaksanaan tugas akhir.

Tahapan pelaksanaan tugas akhir ini adalah sebagai berikut.

\begin{enumerate}
	\item Mempelajari CARLA.
	\item Mempelajari sistem saat ini.
	\item Eksplorasi kakas dan protokol yang akan digunakan.
	\item Perancangan kerangka kerja pengujian.
	\item Implementasi kerangka kerja pengujian.
	\item Eksperimen dan pengujian.
	\item Analisis hasil pengujian.
\end{enumerate}

\section{Jadwal Pelaksanaan Tugas Akhir}

% Tuliskan rencana kegiatan dan jadwal (dirinci sampai per minggu) mulai dari
% awal pelaksanaan Tugas Akhir I s.d. sidang tugas akhir berikut milestones dan
% deliverables yang harus diberikan. Jadwal ini dapat dibantu dengan membuat
% sebuah tabel timeline atau gantt chart.

Berikut adalah jadwal pelaksanaan Tugas Akhir I dan II yang dirinci per minggunya.

\begin{figure}[h]
	\linespread{0.8}
	\resizebox{\textwidth}{!}{
		\begin{ganttchart}[
				y unit title=0.5cm,
				y unit chart=1.3cm,
				bar height=0.7,
				vgrid,
				title height=1,
				bar label font=\tiny,
				bar label node/.style={
						text width=4cm,
						align=right,
						anchor=east,
						font=\raggedleft
					},
			]{1}{20}
			%labels
			\gantttitle{September}{4}
			\gantttitle{Oktober}{4}
			\gantttitle{November}{4}
			\gantttitle{Desember}{4}
			\gantttitle{Januari}{4} \\
			\gantttitlelist{1,...,4}{1}
			\gantttitlelist{1,...,4}{1}
			\gantttitlelist{1,...,4}{1}
			\gantttitlelist{1,...,4}{1}
			\gantttitlelist{1,...,4}{1} \\

			% tasks
			\ganttbar{Menentukan to\-pik \& calon pembimbing}{1}{1} \\
			\ganttbar{Alokasi topik \& pem\-bim\-bing TA oleh tim TA}{2}{2} \\
			\ganttbar{Menyusun Bab II}{3}{6} \\
			\ganttbar{Mempelajari CARLA}{4}{4} \\
			\ganttbar{Mempelajari sistem saat ini}{5}{5} \\
			\ganttbar{Eksplorasi ROS}{6}{7} \\
			\ganttbar{Menyusun Bab I}{7}{10} \\
			\ganttbar{Implementasi jalur komunikasi menggunakan ROS}{8}{11} \\
			\ganttbar{Menyusun Bab III}{11}{13} \\
			\ganttbar{Eksplorasi protokol komunikasi lain}{14}{20} \\
			\ganttbar{Seminar TA I}{17}{18}
		\end{ganttchart}
	}
	\caption{Jadwal Pelaksanaan Tugas Akhir (bagian 1)}
\end{figure}

\begin{figure}[h]
	\begin{center}
		\linespread{0.8}
		\resizebox{\textwidth}{!}{
			\begin{ganttchart}[
					y unit title=0.5cm,
					y unit chart=1.3cm,
					bar height=0.7,
					vgrid,
					title height=1,
					bar label font=\tiny,
					bar label node/.style={
							text width=4cm,
							align=right,
							anchor=east,
							font=\raggedleft
						},
				]{1}{24}
				%labels
				\gantttitle{Februari}{4}
				\gantttitle{Maret}{4}
				\gantttitle{April}{4}
				\gantttitle{Mei}{4}
				\gantttitle{Juni}{4}
				\gantttitle{Juli}{4} \\
				\gantttitlelist{1,...,4}{1}
				\gantttitlelist{1,...,4}{1}
				\gantttitlelist{1,...,4}{1}
				\gantttitlelist{1,...,4}{1}
				\gantttitlelist{1,...,4}{1}
				\gantttitlelist{1,...,4}{1} \\

				% tasks
				\ganttbar{Menyusun Bab IV}{1}{16} \\
				\ganttbar{Perancangan kerangka kerja pengujian}{1}{8} \\
				\ganttbar{Implementasi kerangka kerja pengujian}{9}{16} \\
				\ganttbar{Menyusun Bab V}{17}{18} \\
				\ganttbar{Sidang TA 2}{21}{22}
			\end{ganttchart}
		}
	\end{center}
	\caption{Jadwal Pelaksanaan Tugas Akhir (bagian 2)}
\end{figure}
\chapter{Studi Literatur}

% Bab Studi Literatur digunakan untuk mendeskripsikan kajian literatur yang
% terkait dengan persoalan tugas akhir. Tujuan studi literatur adalah:

% \begin{enumerate}
%     \item menunjukkan kepada pembaca adanya gap seperti pada rumusan masalah
%         yang memang belum terselesaikan,
%     \item memberikan pemahaman yang secukupnya kepada pembaca tentang teori
%         atau pekerjaan terkait yang terkait langsung dengan penyelesaian
%         persoalan, serta
%     \item menyampaikan informasi apa saja yang sudah ditulis/dilaporkan oleh
%         pihak lain (peneliti/Tugas Akhir/Tesis) tentang hasil
%         penelitian/pekerjaan mereka yang sama atau mirip kaitannya dengan
%         persoalan tugas akhir.
% \end{enumerate}
Pada bab ini, Penulis akan menguraikan hasil literatur dalam penyusunan tugas
akhir ini. Subbab pertama membahas simulator CARLA, yaitu perangkat lunak yang
digunakan sebagai alat simulasi. Subbab kedua menjelaskan NVIDIA Pegasus yang
akan digunakan sebagai mesin untuk menjalankan algoritma \textit{decision
making} di lingkungan simulasi dan \textit{production}. Lalu, pada subbab
ketiga akan dibahas beberapa cara komunikasi yang dapat digunakan pada sistem
terdistribusi. Terakhir, subbab keempat akan membahas penelitian-penelitian
terkait simulasi \textit{autonomous vehicle} menggunakan CARLA.

% Sekarang mau ke bab berapa yaaaa.... hmm... ke bab \ref{sec:latarbelakang} ahhhhh. 

\section{Simulator CARLA}
\blindtext

\section{NVIDIA Pegasus}

NVIDIA Pegasus adalah salah satu produk cetusan NVIDIA Corporation di bawah
lini produk NVIDIA Drive PX. Nama pasar dari NVIDIA Pegasus adalah NVIDIA Drive
PX Pegasus. Lini produk NVIDIA Drive sendiri merupakan platform komputer untuk
memberikan fungsionalitas bantuan mengemudi pada kendaraan bermotor.

Mengutip dari \parencite{oh_2017}, NVIDIA Pegasus adalah komputer yang
mendukung pengemudian \textit{autonomous} secara penuh. Artinya, NVIDIA Pegasus
dapat digunakan untuk membuat sebuah kendaraan bermotor menjadi
\textit{autonomous vehcile} jika sensor dan algoritma yang digunakan tepat.

NVIDIA Pegasus menggunakan 2 GPU dengan arsitektur post-Volta dan 2 SoC NVIDIA
Xavier. Kombinasi CPU dan GPU ini dapat menghasilkan 320 TOPS (\textit{trillion
operations/second}) untuk komputasi intelegensi buatan. Untuk koneksi I/O,
NVIDIA Pegasus mendukung sampai dengan 16 kamera (6 di antaranya adalah lidar).

\section{Komunikasi pada Sistem Terdistribusi}

Pada subbab ini akan dibahas beberapa cara komunikasi yang dapat digunakan
untuk menghubungkan dua atau lebih komputer berbeda pada sebuah sistem
terdistribusi. Cara-cara komunikasi tersebut adalah menggunakan Rosbridge, RPC,
\textit{messaging system}, HTTP, dan Websocket.

\subsection{Rosbridge}

Rosbridge adalah protokol komunikasi yang menambahkan antarmuka pada komputer
dengan sistem operasi ROS. Antarmuka yang ditambahkan membuat komputer dapat
\textit{publish} dan \textit{subscribe} ke topik ROS dalam format JSON. Selain
itu, antarmuka tersebut memungkinkan memanggil \textit{service} ROS di antara
hal-hal lainnya.

% TODO: nanya cara ngutip yg bener wkwk
Spesifikasi lengkap untuk rosbridge tertuang di proyek \parencite{ros_bridge}.
Secara arsitektur, protokol rosbridge menggunakan protokol WebSocket untuk
lapisan transpornya. Protokol WebSocket sendiri berjalan di atas protokol TCP
yang artinya protokol rosbridge menjamin data akan sampai dengan urutan yang
benar.

Protokol rosbridge menggunakan format JSON untuk pesannya. Pesan yang valid
harus mengandung \textit{field} \texttt{"op"}. \textit{Field} tersebut
digunakan untuk menentukan jenis pesan. Pesan juga dapat mengandung
\textit{field} \texttt{"id"} yang dapat digunakan sebagai penanda transaksi
atau keterhubungan antara beberapa pesan. Selain kedua \textit{field} tersebut,
pesan rosbridge juga dapat mengandung \textit{field} lainnya tergantung jenis
\texttt{"op"}.

\begin{lstlisting}[language=JSON, caption=contoh pesan valid pada lapisan transpor rosbridge]
{
    "op": "operation"
}
\end{lstlisting}

Jenis pesan yang dapat dikirim dapat dibagi menjadi 3 kategori:
\begin{enumerate}
    \item pesan kompresi atau transformasi,
    \item pesan status rosbridge, dan
    \item pesan operasi.
\end{enumerate}

\subsection{RPC}
\subsection{\textit{messaging system}}
\subsection{HTTP}
\subsection{WebSocket}

\section{Penelitian Terkait}
\blindtext


\chapter{Analisis dan Rancangan Sistem HILS}

Pada bab ini akan dibahas analisis permasalahan pada sistem simulasi yang sudah
ada. Setelah itu, akan dilakukan analisis dan perancangan terhadap solusi yang
akan dibuat pada buku ini.

\section{Deskripsi Umum Proyek \textit{Capstone}}

Tugas akhir yang dikerjakan oleh tim \textit{capstone} bertujuan untuk memenuhi
kebutuhan tim simulasi pada proyek pengembangan trem otonom. Tim simulasi
dibentuk agar pengujian algoritma kendali dan pengumpulan data tidak harus
dilakukan dengan trem yang nyata. Alasannya adalah untuk keamanan, menghemat
biaya, dan menghemat waktu. Pengujian nyata dengan algoritma yang belum siap
dapat menyebabkan orang atau kendaraan lain ditabrak oleh trem. Selain itu,
dengan simulasi para pengembang tidak perlu terjun ke lapangan di Kota Madiun
dan tidak perlu menyewa trem serta rel untuk melakukan pengujian.

Proyek pengembangan trem otonom sendiri sudah memasuki tahun kedua, akan tetapi
pada tahun pertama belum ada tim simulasi. Program untuk simulasi, baik SILS
(\textit{software-in-the-loop-simulation}) maupun HILS
(\textit{hardware-in-the-loop-simulation}), tidak memiliki waktu pengembangan
yang banyak dan hanya sekadar ada. Oleh karenanya, tim \textit{capstone} perlu
mengimplementasikan atau memperbaiki sistem simulasi HILS agar pengujian HILS
dapat  dilakukan. Selain itu, tim \textit{capstone} juga harus dapat membuat
lingkungan simulasi yang semirip mungkin dengan keadaan di Indonesia serta
membuat beberapa skenario simulasi agar pengujian dapat dilakukan secara
otomatis.

Tugas akhir ini bertujuan untuk membuat sistem simulasi agar HILS lebih reliabel
dan memiliki dengan kinerja yang baik. Program HILS sendiri sudah pernah
diimplementasi pada tahun pertama proyek, akan tetapi implementasinya memiliki
beberapa keluhan yang akan dibahas pada bab ini. Dari keluhan-keluhan tersebut
akan dibahas analisis serta rancangan solusi untuk memperbaiki program HILS yang
ada.

\section{Analisis Masalah Sistem HILS Saat ini}

Sistem simulasi yang ada sudah dapat menghubungkan komputer SILS dengan server
komputer RKB/AGX sehingga sistem simulasi HILS sudah dapat digunakan. Akan
tetapi masih ada keluhan terkait kinerja sistem smulasi HILS, yaitu proses
simulasi yang sangat lambat dan menyebabkan simulasi kurang realistis. Jumlah
transaksi data per detik turun dari 4000 transaksi data per detik pada SILS,
turun menjadi 100-110 transaksi data per detik ketika menggunakan HILS dan
layanan web. Target kecepatan simulasi yang harus dicapai untuk dianggap cukup
cepat adalah CARLA dapat berjalan stabil dengan minimum 2 FPS (\textit{frames
per second}).

Padahal kedua komputer pada sistem HILS sudah terhubung pada jaringan lokal
(LAN) yang artinya latensi dan gangguan jaringan akan minimum, jika ada.
Oleh karena itu, kemungkinan \textit{bottleneck} terdapat pada implementasi
mekanisme komunikasi. Implementasi mekanisme komunikasi menggunakan sebuah
layanan web yang arsitekturnya dapat dilihat pada diagram di Gambar
\ref{chapter-2-old-hils}. Proses pada implementasi HILS yang lama dapat dilihat
pada diagram \textit{sequence} di Gambar
\ref{chapter-3-sequence-diagram-old-hils}. Pada proses pengiriman data, terdapat
delapan operasi I/O yang berjalan secara sinkronis, yaitu
\begin{enumerate}
	\item penulisan CARLA \textit{measurement} ke \textit{file} di SILS,
	\item pembacaan CARLA \textit{measurement} dari \textit{file} di SILS,
	\item pengiriman CARLA \textit{measurement} menggunakan HTTP dari SILS ke
		layanan web,
	\item penulisan CARLA \textit{measurement} ke basis data pada layanan web,
	\item permintaan HTTP dari AGX/RKB ke layanan web untuk membaca data,
	\item pembacaan CARLA \textit{measurement} dari basis data pada layanan web,
	\item penulisan CARLA \textit{measurement} ke \textit{file} pada AGX/RKB,
		dan
	\item pembacaan CARLA \textit{measurement} dari \textit{file} pada AGX/RKB.
\end{enumerate}

\begin{figure}[h!]
    \centering
	\includegraphics[width=1.0\textwidth]{resources/chapter-3/sequence-diagram-old-hils-process.png}
	\caption{Proses pengiriman data CARLA \textit{measurement} kondisi saat ini}
    \label{chapter-3-sequence-diagram-old-hils}
\end{figure}

Dari diagram \textit{sequence} dapat diperkiran \textit{bottleneck} disebabkan
banyaknya operasi I/O sinkronis yang berdampak pada \textit{overhead} operasi
I/O. \textit{Overhead} operasi I/O a\-kan menjadi lebih buruk lagi apabila data
yang dikirimkan berukuran besar, misalnya data sensor kamera. Selain itu, sistem
yang ada juga lebih rumit dari seharusnya. Terdapat perantara berupa layanan web
dan basis data padahal data bisa saja dikirimkan langsung dari komputer SILS ke
komputer AGX/RKB.

Selain masalah kinerja, terdapat keluhan juga karena sistem HILS yang ada belum
menggunakan data sensor. Sistem HILS yang ada masih memanfaatkan CARLA
\textit{measurement}. Data dari CARLA \textit{measurement} mencakup posisi x,
posisi y, kecepatan, dan jarak relatif. Data-data ini dibutuhkan untuk
mendapatkan oleh algoritma kendali, akan tetapi seharusnya didapatkan dari
sensor. Pada sistem HILS saat ini, data \textit{measurement} tersebut didapatkan
dengan memanggil fungsi dari API Python CARLA.

\section{Analisis Solusi}

Dari analisis masalah, didapatkan dua keluhan pada implementasi sebelumnya,
yaitu masalah kinerja dan belum ada dukungan terhadap data sensor. Dari
keluhan-keluhan tersebut, dibutuhkan sebuah solusi yang dapat meningkatkan
kinerja HILS dan dapat menggunakan data sensor. Sensor-sensor yang harus
didukung pada solusi adalah sensor kamera, lidar, dan GNSS. Ketiga sensor harus
dapat digunakan secara bersamaan.

Dari kebutuhan-kebutuhan tersebut, ada dua buah alternatif solusi: melakukan
peningkatan (\textit{upgrade}) program HILS yang sudah ada atau menulis ulang
program HILS. Dari kedua alternatif, dipilih alternatif penulisan ulang program
HILS. Alasannya adalah karena pada solusi pertama memiliki kompleksitas yang
lebih tinggi karena ada lebih banyak komponen pada sistem. Selain itu, layanan
web pada solusi pertama tidak terintegrasi langsung pada program utama, baik di
komputer SILS maupun komputer AGX/RKB. Layanan web membutuhkan program bantuan
yang berkomunikasi dengan program utama menggunakan \textit{file}. Hal ini ingin
dihindari karena adanya \textit{overhead} I/O. Selain itu, meskipun layanan web
memiliki \textit{coupling} yang rendah dengan kedua program utama, kohesinya
juga rendah. Karena kedua alasan itulah dinilai alternatif kedua akan lebih
mudah untuk dilaksanakan.

Selanjutnya adalah pemilihan mekanisme komunikasi. Berdasarkan studi literatur,
terdapat dua alternatif untuk mekanisme komunikasi, yaitu ROS dan ZeroMQ. ROS
sendiri adalah salah satu kerangka kerja dan mekanisme komunikasi yang sering
digunakan untuk kebutuhan simulasi robot. Akan tetapi, karena kendala teknis ROS
tidak dapat dipilih. Oleh karena itu, dari pilihan antara ROS dan ZeroMQ,
dipilih ZeroMQ.

Kendala teknis yang menyebabkan ROS tidak dapat digunakan adalah ketidakcocokan
antara versi ROS di komputer SILS dengan komputer AGX/RKB. Komputer SILS
menggunakan sistem operasi Ubuntu 20.04 yang menggunakan ROS 2, sedangkan
komputer AGX/RKB menggunakan sistem operasi Ubuntu 18.04 yang menggunakan ROS 1.
ROS 2 dan ROS 1 memiliki arsitektur komunikasi yang berbeda, hal ini menyebabkan
ketidakcocokan antara ROS 2 dengan ROS 1. Meskipun ada program untuk
menjembatani ROS 2 dan ROS 1, hal tersebut dihindari karena adanya kemungkinan
penambahan latensi.

\section{Rancangan Solusi}

Karena program utama di sisi komputer SILS dan AGX/RKB sudah ada atau sedang
dikerjakan, maka dipililh solusi dalam bentuk pustaka agar pemuatan bisa
dilakukan secara modular dan independen dari kedua program utama. Menciptakan
\textit{coupling} yang rendah antara pustaka dengan kedua program utama. Selain
itu, keuntungan pustaka adalah program GRS jadi dapat memilih untuk memuat
pustaka yang akan dibuat atau tidak pada proses kompilasi sehingga dapat sedikit
menghemat \textit{resource} terutama memori. Pustaka yang dibuat akan disebut
``hils-connector''.

Pustaka yang akan dibuat ada dua, yaitu pustaka C++11 untuk program GRS di
komputer AGX/RKB dan pustaka Python 3 untuk \textit{agent} program
\textit{scenario runner}. Pustaka Python akan disebut ``\textit{producer}''
karena memproduksi data sensor. Pustaka C++ akan disebut ``\textit{consumer}''
karena mengonsumsi data sensor. Selain data sensor, juga akan ada data berupa
kontrol/perintah dengan format sebuah \texttt{int} yang dikirim dari
\textit{consumer} ke \textit{producer} setelah data sensor berhasil diproses.

Lalu untuk fitur pustaka itu sendiri, pustaka akan langsung berkomunikasi satu
sama lain tanpa menggunakan perantara. Hal ini membuat solusi lebih sederhana
dan mengurangi \textit{overhead} untuk berkomunikasi dengan perantara. Pustaka
juga akan menyediakan API untuk mengirimkan data dan menerima data. Untuk
mendukung berbagai jenis sensor, pustaka juga akan memiliki fitur
\textit{parsing}, serialisasi, dan deserialisasi untuk sensor GNSS, kamera, dan
lidar.

API yang disediakan oleh pustaka ``producer'' akan langsung mengkonsumsi data
sensor dari CARLA sehingga pengguna pustaka tidak perlu melakukan modifikasi
apapun pada data sensor CARLA. Begitu juga pada pustaka ``consumer''. Balikan
API akan disesuikan sehingga pengguna pustaka dapat langsung menggunakan data
sensor dan langsung ``memasukkannya'' ke sensor virtual NVIDIA DriveWorks.

Lalu, untuk mengurangi berbagai \textit{overhead} tambahan yang dapat muncul
karena operasi pada jaringan, dipilih mekanisme komunikasi menggunakan ZeroMQ.
ZeroMQ adalah \textit{message queue} sehingga cocok digunakan untuk operasi
asinkron yang sesuai dengan pengiriman data sensor dari CARLA. Lalu, ZeroMQ juga
sangat dekat dengan TCP, tapi tanpa kompleksitas \textit{raw} TCP. Hal tersebut
dikarenakan salah satu tujuan utama ZeroMQ adalah mengurangi latensi hinga
sesedikit mungkin dan memaksimalkan \textit{throughput}. ZeroMQ juga tidak
memiliki \textit{broker}, hal ini sesuai dengan keinginan menghilangi perantara
sehingga \textit{producer} dan \textit{consumer} dapat langsung berkomunikasi
satu sama lain.

Dari solusi pustaka yang ditawarkan, dapat dibentuk sebuah arsitektur sistem
HILS. Arsitektur ini dapat dilihat pada diagram \textit{deployment} di Gambar
\ref{chapter-3-new-architecture}. Lalu, gambaran kasar proses pada sistem
simulasi untuk satu \textit{step} simulasi dapat dilihat pada diagram
\textit{sequence} di Gambar \ref{chapter-3-new-sequence}. Perlu dicatat bahwa
HILS \textit{agent} adalah \textit{agent} yang digunakan pada program
ScenarioRunner.

\begin{figure}[h!]
    \centering
    \includegraphics[width=1.0\textwidth]{resources/chapter-3/deployment-diagram-new-hils.png}
    \caption{Arsitektur Sistem HILS Baru}
    \label{chapter-3-new-architecture}
\end{figure}

\begin{figure}[h!]
    \centering
    \includegraphics[width=1.0\textwidth]{resources/chapter-3/sequence-diagram-new-hils-kasar.png}
		\caption{Gambaran kasar proses satu \textit{step} simulasi}
    \label{chapter-3-new-sequence}
\end{figure}

\chapter{Implementasi dan Pengujian Sistem HILS}\label{chapter-4}

\section{Implementasi Sistem HILS}

Berdasarkan analisis dan rancangan solusi pada Bab \ref{chapter-3}, dapat
dimulai implementasi sistem HILS baru. Implementasi sistem HILS dimulai dengan
eksplorasi ROS2 dan ZeroMQ. Kemudian dilanjutkan dengan pembuatan program
\textit{proof of concept} (POC) menggunakan salah satu mekanisme komunikasi.
POC dibuat untuk menunjukan bahwa mekanisme komunikasi yang digunakan dapat
melakukan transfer data kamera sehingga CARLA berjalan dengan setidaknya 2 FPS.
Setelah POC diterima oleh ketua tim simulasi, dilanjutkan penulisan pustaka
\textit{consumer} dan terakihir implementasi pustaka \textit{producer}.

Dari proses eksplorasi dan implementasi POC, didapatkan metode komunikasi yang
cocok adalah ZeroMQ. ROS 2 sendiri gagal pada tahap POC dikarenakan sistem
operasi tidak kompatibel dengan ROS 2. Oleh karena itu, implementasi pustaka
akan menggunakan ZeroMQ. Sedangkan, ROS 2 tidak akan digunakan lagi pada tugas
akhir ini.

Pustaka yang pertama ditulis adalah pustaka \textit{consumer} (sisi
komputer AGX/RKB) karena program utama komputer SILS (pengguna
\textit{producer}) belum selesai pada saat proses penulisan pustaka. Akibatnya,
pengujian pustaka \textit{consumer} lebih mudah dilakukan pada saat itu.

Pustaka \textit{producer} dan \textit{consumer} akan memanfaatkan pemrograman
berorientasi objek untuk menstruktur kodenya. Pustaka yang dibuat juga dibuat
seabstrak mungkin dan tidak \textit{coupled} pada trem saja. Sehingga pustaka
yang ditulis dapat digunakan untuk simulasi \textit{hardware-in-the-loop} jenis
kendaraan otonom lainnya.

Diagram kelas dari pustaka \textit{consumer} dapat dilihat pada gambar
\ref{chapter-4-consumer-class-diagram}. Kelas yang melakukan komunikasi adalah
kelas abstrak \texttt{Endpoint}. Kelas tersebut dibuat abstrak dan diinstansiasi
menggunakan pola pemrograman \textit{factory}. \texttt{Endpoint} dibuat abstrak
agar apabila ingin ditambahkan metode komunikasi lain, hal tersebut dapat
dilakukan dengan mudah. Contoh kasus penggunaan penambahan metode komunikasi
lain adalah pengujian atau \textit{benchmarking} metode komunikasi. Selain itu,
terdapat kelas yang akan menyediakan layanan untuk program GRS, yaitu
\texttt{CarlaService}. Kelas ini mengabstraksi komunikasi dan deserialisasi
ataupun serialisasi data dari \texttt{Endpoint}.
\begin{figure}[!htbp]
	\centering
	\includegraphics[width=1.0\textwidth]{resources/chapter-4/consumer-class_diagram.png}
	\caption{Diagram Kelas Pustaka \textit{Consumer}}
	\label{chapter-4-consumer-class-diagram}
\end{figure}

Setelah pustaka \textit{consumer} berhasil, implementasi dilanjutkan dengan
pembuatan pustaka \textit{producer}. Diagram kelas pustaka \textit{producer}
dapat dilihat pada gambar \ref{chapter-4-producer-class-diagram}. Pembuatan
pustaka \textit{producer} juga mengikuti filosofi penulisan pustaka
\textit{consumer}. Pustaka dibuat seabstrak mungkin sehingga tidak
\textit{coupled} dengan trem. Kelas \texttt{Endpoint} juga dibuat abstrak agar
dapat ditambahkan metode komunikasi yang lain. Perbedaan implementasi adalah
pada kelas yang berinteraksi dengan program utama. Pada pustaka
\textit{producer}, ada empat kelas yang berinteraksi dengan program utama, yaitu
\texttt{CameraHandler}, \texttt{LidarHandler}, \texttt{GnssHandler}, dan
\texttt{ControlHandler}. Keempat kelas memiliki peran masing-masing dan
terspesialisasi untuk menangani data dari sensor virtual CARLA tertentu.
\begin{figure}[!htbp]
	\centering
	\includegraphics[width=1.0\textwidth]{resources/chapter-4/producer-class_diagram.png}
	\caption{Diagram Kelas Pustaka \textit{Producer}}
	\label{chapter-4-producer-class-diagram}
\end{figure}

Setelah kedua pustaka diimplementasi, dilakukan pengujian pustaka dengan program
kecil. Lalu, kedua pustaka diintegrasi dengan kedua program utama. Setelah itu,
dapat dianggap sistem implementasi HILS yang baru sudah diimplementasi dan
pengujian HILS dapat dilakukan. Akan tetapi untuk kebutuhan tugas akhir, sebelum
pengujian HILS dilakukan, akan dilaksanakan pengujian. Metode pengujian dan
aspek sistem yang diuji akan dibahas pada Subbab
\ref{chapter-4-testing-methodology}.

\section{Metode Pengujian Implementasi dan
  Kinerja}\label{chapter-4-testing-methodology}

Pengujian akan menguji dua aspek, yaitu
\begin{enumerate}
	\item pengujian implementasi: meninjau kemampuan sistem HILS dalam mengirim,
	      menerima, dan menggunakan data dari sensor; dan
	\item pengujian kinerja: meninjau latensi yang dibutuhkan untuk mengirim
	      data.
\end{enumerate}
Bagian ini, Subbab \ref{chapter-4-testing-methodology}, akan membahas metode dan
rencana pengujian kedua aspek tersebut. Pengujian dilakukan dengan menjalankan
beberapa skenario simulasi lalu dibandingkan dengan kriteria kedua aspek. Sebuah
aspek dinyatakan berhasil/lolos apabila semua kriterianya berhasil dicapai.

\subsection{Pengujian Implementasi Sistem HILS}

Pengujian implementasi dilakukan dengan menjalankan program utama di komputer
SILS dan komputer AGX/RKB. Kemudian, akan dilakukan observasi untuk memeriksa
sistem HILS sudah memenuhi kriteria-kriteria pada tabel
\ref{chapter-4-tbl-impl-criteria} atau belum.
\begin{table}[!htbp]
	\begin{center}
		\begin{tabular}{|l|l|}
			\hline
			\textbf{Kode} & \textbf{Deskripsi}                                     \\
			\hline
			IMPL-01       & Kecepatan trem di program GRS sesuai dengan bacaan
			sensor                                                                 \\
			              & GNSS.                                                  \\
			\hline
			IMPL-02       & Tampilan kamera di program GRS sesuai dengan yang
			ada di                                                                 \\
			              & program ScenarioRunner.                                \\
			\hline
			IMPL-03       & Tampilan lidar di program GRS menggambarkan
			objek-objek                                                            \\
			              & yang ada di sekitar trem.                              \\
			\hline
			IMPL-04       & Trem dapat maju atau berhenti tanpa masukan dari
			papan ketik.                                                           \\
			\hline
			IMPL-05       & Trem maju ketika mendapatkan kendali maju dan berhenti \\
			              & ketika mendapatkan kendali berhenti.                   \\
			\hline
		\end{tabular}
	\end{center}

	\caption{kriteria pengujian implementasi sistem HILS}
	\label{chapter-4-tbl-impl-criteria}
\end{table}

Kriteria ini selaras dengan tujuan tugas akhir yang kedua, yaitu
mengimplementasikan sistem simulasi yang dapat mengirimkan, menerima, dan
memanfaatkan data dari berbagai jenis sensor.

\subsection{Pengujian Kinerja Sistem HILS}

Dari segi kinerja, hal yang ingin dipastikan adalah CARLA dapat berjalan stabil
dengan kecepatan minimum 2 FPS (\textit{frames per second}). Hal tersebut
diobservasi dengan menjalankan kedua program utama. Selain dari kecepatan
simulator, aspek kinerja juga akan dinilai dari perbandingan dengan sistem HILS
yang ada. Latensi pengiriman data harus lebih rendah dibandingkan latensi sistem
HILS yang ada.

Kedua kriteria tersebut dituangkan pada tabel \ref{chapter-4-tbl-perf-criteria}.
\begin{table}[!htbp]
	\begin{center}
		\begin{tabular}{|l|l|}
			\hline
			\textbf{Kode} & \textbf{Deskripsi}                                       \\
			\hline
			PERF-01       & CARLA dapat berjalan dengan stabil dengan kecepatan      \\
			              & minimum 2 FPS.                                           \\
			\hline
			PERF-02       & Latensi pengiriman data lebih rendah dibandingkan sistem \\
			              & HILS yang ada.                                           \\
			\hline
		\end{tabular}
	\end{center}
	\caption{kriteria pengujian kinerja sistem HILS}
	\label{chapter-4-tbl-perf-criteria}
\end{table}

Untuk kriteria kedua, pengujian latensi implementasi HILS sebelumnya akan
dilakukan secara teoretis. Hal ini karena implementasi HILS sebelumnya sudah
sulit untuk dijalankan. Selain itu, jenis data yang dikirim pada implementasi
HILS sebelumnya juga berbeda. Pengujian secara teoretis ini dilakukan dengan
menulis ulang sebagian dari mekanisme komunikasi implementasi HILS sebelumnya.
Bagian yang akan ditulis ulang adalah operasi pembacaan data dari \textit{file},
penulisan dan pembacaan ke basis data, serta penulisan data sensor yang dibaca
dari basis data ke \textit{file}. Dengan demikian, ada 4 operasi I/O dari
implementasi HILS sebelumnya yang ditulis ulang untuk pengujian latensi secara
teoretis.

Apabila sistem HILS berhasil memenuhi kedua kriteria tersebut, sistem dapat
dianggap sudah berhasil memenuhi kedua tujuan tugas akhir. Hal tersebut karena
kinerja sistem akan dipengaruhi mekanisme komunikasi yang digunakan (tujuan
pertama) dan juga cara pustaka di sistem HILS menserialisasi (mengirim) dan
mendeserialisasi (menerima) data sensor (tujuan kedua).

\section{Hasil Pengujian}

Penjabaran hasil pengujian aspek implementasi dan kinerja juga akan dipisah.
Penjabaran dipisah karena hal yang diujikan dan dipastikan dari kedua aspek
berbeda.

\subsection{Hasil Pengujian Implementasi Sistem HILS}

Dari pengujian yang dilakukan, ditemukan bahwa sistem HILS dapat memenuhi
seluruh kriteria yang untuk pengujian implementasi sistem. Program GRS berhasil
mengonsumsi data-data sensor GNSS, kamera, dan lidar. Kemudian, data ketiga
sensor tersebut dapat ditampilkan dengan tepat pada program GRS.

\subsection{Hasil Pengujian Kinerja Sistem HILS}

Dari pengujian kinerja, ditemukan sistem gagal memiliki kinerja yang buruk
apabila sensor lidar digunakan. Kecepatan simulator bahkan tidak mencapai 1 FPS.

Akan tetapi ketika diujicobakan lagi tanpa sensor lidar, sistem memiliki kinerja
yang sangat baik. Untungnya, ada versi dari program GRS yang tidak memanfaatkan
sensor lidar sehingga program GRS versi tersebut digunakan untuk pengujian
kinerja selanjutnya.

Ketika diujikan dengan program GRS tanpa lidar, latensi pengiriman data sensor
GNSS dan kamera tidak terasa ketika simulasi dijalankan. Proses pengiriman data
sensor kamera dan GNSS terasa instan ketika simulasi dijalankan.

Lalu, ketika dibandingkan dengan sistem sebelumnya berikut adalah data-datanya.

\section{Pembahasan}
\blindtext

\chapter{Penutup}\label{chapter-5}

Pada bab ini akan dijelaskan kesimpulan dari tugas akhir ini serta saran untuk
pengembangan sistem HILS kedepannya dan saran untuk proyek \textit{capstone}.

\section{Kesimpulan}

Dari tugas akhir ini ada beberapa kesimpulan yang dapat dibuat, yaitu
\begin{enumerate}
	\item berhasil dilakukan penulisan ulang sistem HILS,
	\item sistem HILS yang baru sudah dapat memanfaatkan data sensor virtual
	dari CARLA,
	\item trem di CARLA berhasil dikendalikan oleh program GRS menggunakan
	komputer AGX (NVIDIA Pegasus) pada sistem HILS baru, dan
	\item latensi dan kinerja pada sistem HILS baru setidaknya secara rata-rata
	2,5 kali lebih cepat jika dibandingkan dengan sistem HILS sebelumnya.
\end{enumerate}

\section{Saran}

Saran untuk pengembangan sistem HILS adalah sebagai berikut.
\begin{enumerate}
	\item a
\end{enumerate}

Berikut adalah saran untuk proyek \textit{capstone} \capstoneTitle.
\begin{enumerate}
	\item a
\end{enumerate}

%----------------------------------------------------------------%

% Daftar pustaka
\printbibliography

% Setting judul lampiran
\titlespacing*{\chapter}{0pt}{0pt}{0pt}
\titlespacing*{\section}{0pt}{0pt}{*1}

% Setting judul anak lampiran
\titleformat*{\section}{\bfseries}

% Index
\appendix
\chapter{Data Kinerja Perhitungan RTT}\label{appendix-performance-data}
% TODO: find a better way to set counter
\setcounter{section}{0}

Berkas \texttt{CSV} berisi data yang digunakan dalam perhitungan RTT dapat
diakses pada pranala-pranala berikut.

\section{Data RTT Mentah (Sebelum Dikurangi Waktu Pemrosesan)}

\begin{enumerate}
	\item data pertama:
	      \url{https://github.com/jspmarc/final-project-data/blob/master/data/new-hils/2023-07-22_001/log_delta_time_raw_rtt.csv}
	\item data kedua:
	      \url{https://github.com/jspmarc/final-project-data/blob/master/data/new-hils/2023-07-22_002/log_delta_time_raw_rtt.csv}
\end{enumerate}

\section{Data Waktu Pemrosesan}

\begin{enumerate}
	\item data pertama:
	      \url{https://github.com/jspmarc/final-project-data/blob/master/data/new-hils/2023-07-22_001/log_delta_time_process_latency.csv}
	\item data kedua:
	      \url{https://github.com/jspmarc/final-project-data/blob/master/data/new-hils/2023-07-22_002/log_delta_time_process_latency.csv}
\end{enumerate}

% TODO: find a better way to set counter
\includepdf[pages=1,scale=.7,pagecommand={\chapter{Program Pemrosesan
			  Data}\label{appendix-algoritma-monitoring}\setcounter{section}{0}Kode Jupyter
			Notebook juga dapat diakses pada pranala
			\url{https://github.com/jspmarc/final-project-data/blob/master/data-processor.ipynb}.},
	linktodoc=false]{resources/appendix-2-data-processor.pdf}
\includepdf[pages=2-,scale=.7,pagecommand={},linktodoc=false]{resources/appendix-2-data-processor.pdf}

% \chapter{Rencana Umum Proyek}
% TODO: find a better way to set counter
\setcounter{section}{1}


\end{document}
