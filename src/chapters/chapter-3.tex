\chapter{Analisis dan Rancangan Perbaikan Sistem Simulasi}

Pada bab ini akan dibahas analisis permasalahan pada sistem simulasi yang sudah
ada. Setelah itu, akan dilakukan analisis terhadap solusi yang akan dibahas pada
juga pada bab ini. Lalu, akan dibahas perancangan perbaikan sistem simulasi.

\section{Analisis Masalah}

Sistem simulasi yang ada sudah dapat menghubungkan server CARLA dengan server
NVIDIA Pegasus. Kedua server dihubungkan di sebuah jaringan menggunakan sebuah
layanan web. Layanan web ini memanfaatkan teknologi HTTP.

Solusi yang memanfaatkan layanan web dan HTTP ini merupakan \textit{bottleneck}
besar pada sistem simulasi. Kinerja yang diberikan pada arsitektur sekarang
sangat buruk. Dari 4000 transaksi per detik pada SILS, ketika menggunakan HILS
dan layanan web, turun menjadi 100-110 transaksi per detik. Kinerja buruk ini
mengakibatkan sistem simulasi terhambat pada jalur komunikasinya.

Selain itu, teknologi HTTP di bawah HTTP 2.0 juga tidak mendukung data
\textit{binary} dengan baik. Hal ini mengakibatkan data dari sensor membutuhkan
waktu dan \textit{resource} lebih untuk pemrosesannya.

Berdasarkan analisis masalah, dibutuhkan pembaruan pada sistem simulasi agar
memiliki kinerja yang lebih baik. Selain itu, diperlukan jalur komunikasi yang
mendukung pengiriman data \textit{binary}. Selain cepat dan mendukung
\textit{binary}, jalur komunikasi tentu saja harus reliabel agar tidak ada data
yang hilang.

\section{Analisis Perbaikan Sistem Simulasi}

% Jelasin bentuk sistemnya yang ideal gimana
\blindtext

\section{Rancangan Pebaikan Sistem Simulasi}

% Jelasin desain arsitektur sistem, pilihan protokol, cara milih protokol
\blindtext