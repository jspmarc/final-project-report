\chapter{Implementasi dan Pengujian Sistem HILS}

\section{Implementasi Pustaka untuk Sistem HILS}

Berdasarkan analisis dan rancangan solusi pada bab 3, dapat dimulai implementasi
sistem HILS baru. Implementasi sistem HILS dimulai dengan eksplorasi ROS2 dan
ZeroMQ. Kemudian dilanjutkan dengan pembuatan program \textit{proof of concept}
menggunakan salah satu mekanisme komunikasi. Setelah \textit{proof of concept}
diterima, dilanjutkan penulisan pustaka \textit{consumer} dan terakihir
implementasi pustaka \textit{producer}.

Melalui proses eksplorasi dan penulisan \textit{proof of concept}, didapatkan
metode komunikasi yang cocok adalah ZeroMQ. Oleh karena itu, proses implementasi
dilanjutkan dengan fokus pada ZeroMQ.

Pustaka yang pertama ditulis adalah pustaka \textit{consumer} (si\-si
kom\-pu\-ter AGX/RKB) karena program utama komputer SILS (pengguna
\textit{producer}) belum selesai pada proses penulisan pustaka. Akibatnya,
pengujian pustaka \textit{consumer} lebih mudah dilakukan pada saat itu.

Pustaka \textit{producer} dan \textit{consumer} akan memanfaatkan pemrograman
berorientasi objek untuk menstruktur kodenya. Pustaka yang dibuat juga dibuat
seabstrak mungkin dan tidak \textit{coupled} pada trem saja. Sehingga pustaka
yang ditulis dapat digunakan untuk simulasi \textit{hardware-in-the-loop} jenis
kendaraan otonom lainnya.

Diagram kelas dari pustaka \textit{consumer} dapat dilihat pada gambar
\ref{chapter-4-consumer-class-diagram}. Kelas (atau \textit{interface}) yang
melakukan komunikasi adalah kelas abstrak \texttt{Endpoint}. Kelas tersebut
dibuat abstrak dan diinstansiasi menggunakan pola pemrograman \textit{factory}.
\texttt{Endpoint} dibuat abstrak agar apabila ingin ditambahkan metode
komunikasi lain, hal tersebut dapat dilakukan dengan mudah. Contoh kasus
penggunaan penambahan metode komunikasi lain adalah pengujian atau
\textit{benchmarking} metode komunikasi. Selain itu, terdapat kelas yang akan
menyediakan layanan untuk program GRS, yaitu \texttt{CarlaService}. Kelas ini
mengabstraksi komunikasi dan deserialisasi ataupun serialisasi data dari
\texttt{Endpoint}.

\begin{figure}[h!]
    \centering
    \includegraphics[width=1.0\textwidth]{resources/chapter-4/consumer-class_diagram.png}
	\caption{Diagram Kelas Pustaka \textit{Consumer}}
    \label{chapter-4-consumer-class-diagram}
\end{figure}

Setelah pustaka \textit{consumer} berhasil, implementasi dilanjutkan dengan
pembuatan pustaka \textit{producer}. Diagram kelas pustaka \textit{producer}
dapat dilihat pada gambar \ref{chapter-4-producer-class-diagram}. Pembuatan
pustaka \textit{producer} juga mengikuti filosofi penulilsan pustaka
\textit{consumer}. Pustaka dibuat seabstrak mungkin sehingga tidak
\textit{coupled} dengan trem. Selain itu, antarmuka \texttt{Endpoint} juga
dibuat abstrak agar dapat ditambahkan metode komunikasi yang lain. Perbedaan
implementasi adalah pada kelas yang berinteraksi dengan program utama. Pada
pustaka \textit{producer}, ada empat kelas yang berinteraksi dengan program
utama, yaitu \texttt{CameraHandler}, \texttt{LidarHandler},
\texttt{GnssHandler}, dan \texttt{ControlHandler}. Keempat kelas memiliki peran
masing-masing dan terspesialisasi untuk menangani data atau sensor tertentu.

\begin{figure}[h!]
    \centering
    \includegraphics[width=1.0\textwidth]{resources/chapter-4/producer-class_diagram.png}
	\caption{Diagram Kelas Pustaka \textit{Producer}}
    \label{chapter-4-producer-class-diagram}
\end{figure}

\section{Metode Pengujian Implementasi dan Kinerja}

Pengujian akan menguji dua aspek, yaitu
\begin{itemize}
	\item pengujian implementasi: meninjau kemampuan sistem HILS dalam mengirim,
		menerima, dan menggunakan data dari sensor, dan
	\item pengujian kinerja: meninjau latensi yang dibutuhkan untuk mengirim
		data.
\end{itemize}
Bagian ini akan membahas metode dan rencana pengujian kedua aspek tersebut.
Pengujian yang dilakukan langsung menguji sistem dan tidak hanya menguji pustaka
karena pada sistem HILS sudah termasuk pustaka yang dibuat dan hasil akhir
tugas akhir ini adalah sebuah sistem HILS yang memanfaatkan suatu pustaka untuk
komunikasinya.

\subsection{Pengujian Implementasi Sistem HILS}

Pengujian implementasi dilakukan dengan menjalankan program utama di komputer
SILS dan komputer AGX/RKB. Kemudian, akan dilakukan observasi untuk memeriksa
sistem HILS sudah memenuhi kriteria-kriteria berikut atau belum.

\begin{itemize}
	\item kecepatan trem di program GRS sesuai dengan bacaan sensor GNSS,
	\item tampilan kamera di program GRS sesuai dengan yang ada di program
		ScenarioRunner,
	\item tampilan lidar di program GRS menggambarkan objek-objek yang ada di
		sekitar trem,
	\item trem dapat maju atau berhenti tanpa masukan dari papan ketik, dan
	\item trem maju ketika mendapatkan kendali maju dan berhenti ketika
		mendapatkan kendali berhenti.
\end{itemize}

Kriteria ini selaras dengan tujuan tugas akhir yang kedua, yaitu
mengimplementasikan sistem simulasi yang dapat mengirimkan, menerima, dan
memanfaatkan data dari berbagai jenis sensor.

\subsection{Pengujian Kinerja Sistem HILS}

Dari segi kinerja, hal yang ingin dipastikan adalah CARLA dapat berjalan dengan
stabil dan kecepatan minimum 2 FPS (\textit{frames per second}). Hal tersebut
diobservasi dengan menjalankan kedua program utama. Selain dari kecepatan
simulator, aspek kinerja juga akan dinilai dari perbandingan dengan implementasi
HILS sebelumnya. Latensi pengiriman data harus lebih rendah dibandingkan latensi
implementasi HILS sebelumnya.

Untuk kriteria kedua, pengujian latensi implementasi HILS sebelumnya akan
dilakukan secara teoretis. Hal ini karena implementasi HILS sebelumnya sudah
sulit untuk dijalankan. Selain itu, jenis data yang dikirim pada implementasi
HILS sebelumnya juga berbeda. Pengujian secara teoretis ini dilakukan dengan
menulis ulang sebagian dari mekanisme komunikasi implementasi HILS sebelumnya.
Bagian yang akan ditulis ulang adalah operasi pembacaan data dari \textit{file},
penulisan dan pembacaan ke basis data, serta penulisan data sensor yang dibaca
dari basis data ke \textit{file}. Dengan demikian, ada 4 operasi I/O dari
implementasi HILS sebelumnya yang ditulis ulang untuk pengujian latensi secara
teoretis.

Apabila sistem HILS berhasil memenuhi kedua kriteria tersebut, sistem dapat
dianggap sudah berhasil memenuhi kedua tujuan tugas akhir. Hal tersebut karena
kinerja sistem akan dipengaruhi mekanisme komunikasi yang digunakan (tujuan
pertama) dan juga cara pustaka di sistem HILS menserialisasi (mengirim) dan
mendeserialisasi (menerima) data sensor (tujuan kedua).

\section{Hasil Pengujian}

Penjabaran hasil pengujian aspek implementasi dan kinerja juga akan dipisah.
Penjabaran dipisah karena hal yang diujikan dan dipastikan dari kedua aspek
berbeda.

\subsection{Hasil Pengujian Implementasi Sistem HILS}

Dari pengujian yang dilakukan, ditemukan bahwa sistem HILS dapat memenuhi
seluruh kriteria yang untuk pengujian implementasi sistem. Program GRS berhasil
mengonsumsi data-data sensor GNSS, kamera, dan lidar. Kemudian, data ketiga
sensor tersebut dapat ditampilkan dengan tepat pada program GRS.

\subsection{Hasil Pengujian Kinerja Sistem HILS}

Dari pengujian kinerja, ditemukan sistem gagal memiliki kinerja yang buruk
apabila sensor lidar digunakan. Kecepatan simulator bahkan tidak mencapai 1 FPS.

Akan tetapi ketika diujicobakan lagi tanpa sensor lidar, sistem memiliki kinerja
yang sangat baik. Untungnya, ada versi dari program GRS yang tidak memanfaatkan
sensor lidar sehingga program GRS versi tersebut digunakan untuk pengujian
kinerja selanjutnya.

Ketika diujikan dengan program GRS tanpa lidar, latensi pengiriman data sensor
GNSS dan kamera tidak terasa ketika simulasi dijalankan. Proses pengiriman data
sensor kamera dan GNSS terasa instan ketika simulasi dijalankan.

Lalu, ketika dibandingkan dengan sistem sebelumnya berikut adalah data-datanya.

\section{Pembahasan}
\blindtext
