\chapter{Penutup}\label{chapter-5}

Pada bab ini akan dijelaskan kesimpulan dari tugas akhir ini serta saran untuk
pengembangan sistem HILS kedepannya dan saran untuk proyek \textit{capstone}.

\section{Kesimpulan}

Dari tugas akhir ini ada beberapa kesimpulan yang dapat dibuat, yaitu
\begin{enumerate}
	\item berhasil dilakukan penulisan ulang mekanisme komunikasi sistem HILS
	      dalam bentuk pustaka;
	\item sistem HILS yang baru sudah dapat memanfaatkan data sensor virtual
	      dari CARLA serta trem di CARLA berhasil dikendalikan oleh program GRS
	      menggunakan komputer AGX (NVIDIA Pegasus) pada sistem HILS baru, dan;
	\item latensi dan kinerja pada sistem HILS dengan mekanisme komunikasi baru
	      setidaknya secara rata-rata 2,5 kali lebih cepat jika dibandingkan
	      dengan sistem HILS sebelumnya.
\end{enumerate}

\section{Saran}

Saran untuk pengembangan pustaka dan mekanisme komunikasi adalah sebagai
berikut.
\begin{enumerate}
	\item perbaikan kode pustaka agar dapat menggunakan sensor lidar virtual,
	\item pengujian dengan multi-kamera untuk menguji sistem dapat menggunakan
	      agar lebih sesuai dengan dunia nyata,
	\item pengujian dengan berbagai mekanisme/protokol komunikasi lain untuk
	      mencoba mengurangi \textit{overhead} kinerja akibat jaringan, dan
	\item mengurangi jumlah kelas yang berinteraksi langsung dengan program
	      ScenarioRunner dari empat menjadi satu kelas.
	      % \item membuat penerimaan data di sisi program GRS terjadi secara paralel.
\end{enumerate}

Berikut adalah saran untuk proyek \textit{capstone} pengembangan trem.
\begin{enumerate}
	\item memperbaiki kode program GRS karena sangat sulit untuk dibaca dan
	      dimodifikasi, dan
	\item menambahkan \textit{code review}, \textit{static code analysis}, dan
	      Git \textit{flow} untuk program GRS untuk menjaga kualitas dan keamanan
	      kode.
\end{enumerate}
