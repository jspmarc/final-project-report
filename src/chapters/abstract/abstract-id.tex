\chapter*{ABSTRAK}
\addcontentsline{toc}{chapter}{\MakeUppercase{Abstrak}}

%taruh abstrak bahasa indonesia di sini
\begin{center}
	\bfseries \MakeUppercase{\thetitle}

	\normalfont\normalsize
	Oleh:

	\theauthor
\end{center}

\begin{singlespace}
	Dengan perkembangan pesat kota-kota di Indonesia dan semakin besarnya
	kebutuhan untuk mengurangi masalah lalu lintas dan masalah lingkungan,
	dibutuhkan transportasi umum yang efisien dalam membawa penumpang, aman, dan
	murah. Salah satu transportasi ini adalah trem otonom. Trem otonom tidak
	membutuhkan masinis dalam operasinya. Artinya, biaya operasional trem dapat
	lebih menjadi murah dan keamanan dapat lebih dijamin.

	Akan tetapi, pengembangan trem otonom ini membutuhkan biaya yang banyak dan
	waktu yang lama jika dilakukan secara langsung. Oleh karena itu digunakan
	sebuah sistem simulasi yang dapat menguji perangkat keras dan
	perangkat lunak yang digunakan pada trem otonom nanti. Skema simulasi
	tersebut dapat disebut juga dengan \textit{hardware-in-the-loop-simulation}
	(HILS). Pada sistem HILS, dimanfaatkan simulator CARLA untuk menjalankan
	dunia virtual dan mendapatkan data dari sensor virtual.

	Pada proyek pengembangan trem otonom ini, sudah ada sistem HILS. Akan
	tetapi, belum ideal karena kinerja yang buruk dan tidak dapat menggunakan
	data sensor. Kinerja buruk ini disebabkan banyaknya operasi I/O dan
	arsitektur sistem yang lebih kompleks dari seharusnya. Oleh karena itu,
	telah dibuat sebuah mekanisme komunikasi baru untuk sistem HILS demi
	menyelesaikan kedua masalah tersebut.

	Mekanisme komunikasi baru yang diimplementasikan berhasil memiliki latensi
	setidaknya 2,5x lebih cepat dari mekanisme komunikasi lama. Selain itu,
	simulasi berhasil menggunakan data sensor virtual dan simulator CARLA
	berhasil berjalan dengan lebih dari 2 FPS.

	\textbf{\textit{Kata kunci: komunikasi HILS, CARLA untuk HILS, simulasi
			kendaraan otonom}}
\end{singlespace}
\clearpage
