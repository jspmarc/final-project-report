\clearpage
\chapter*{Abstract}
\addcontentsline{toc}{chapter}{\MakeUppercase{Abstract}}

%put your abstract here
\begin{center}
	\bfseries \MakeUppercase{\thetitleEn}

	\normalfont\normalsize
	By:

	\theauthor
\end{center}

\begin{singlespace}
	With the rapid growth of cities in Indonesia and the need to reduce impacts
	of growth to traffic and the environment, a public transport that can
	efficiently carry passengers, safe, and inexpensive.  One of these public
	transport is autonomous tram. The autonomous tram doesn't need a driver to
	operate it. This results in cheaper operational cost and safety can be
	better guaranteed.

	However, the development of autonomous tram is expensive and needs a lot of
	time if the testings were done in a real-world environment. Hence, a
	simulation system that an test the hardware and software of the autonomous
	tram is used for testing and development. A simulation that tests the
	hardware and software is also known as hardware-in-the-loop-simulation
	(HILS). In the HILS system, the CARLA simulator is used to run the virtual
	world and to get data from virtual sensors.

	In the autonomous tram development project, a HILS system alrady exists.
	However, the system is not yet ideal since its performance is bad and sensor
	data can't be used in the simulation system. The bad performance is caused
	by heaps of I/O operations for communication and the system's architecture
	which is more complicated than it needs to be. Therfore, a new communication
	mechanism is created for the HILS system to solve the bad performance and
	inability to use sensor data.

	The new communication mechanism is at least 2.5x faster when compared to the
	old HILS system's communication mechanism. Moreover, the simulation system
	can use sensor data  while keeping CARLA running with at least 2 FPS.

	\textbf{\textit{Keywords: HILS communication, CARLA for HILS, autonomous
			vehicle simulation}}
\end{singlespace}

\clearpage
