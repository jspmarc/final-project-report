\chapter{Pendahuluan}

\section{Latar Belakang}

Dalam pengembangan kota-kota di Indonesia, dibutuhkan kendaraan umum yang aman,
murah, mudah diakses, dan berkelanjutan (\textit{sustainable}). Salah satu moda
transportasi umum yang dapat memenuhi keempat hal tersebut adalah trem listrik.
Agar bisa lebih aman, trem listrik akan memanfaatkan inteligensi buatan dan
pembelajaran mesin sehingga trem tidak membutuhkan masinis dan bisa beroperasi
secara otonom.

Pengembangan teknologi pembelajaran mesin dan algoritma kendali otonom yang
digunakan memerlukan banyak data dan pengujian. Proses pengumpulan data dan
pengujian ini akan mahal dan membutuhkan waktu yang lama. Selain itu, akan
berbahaya apabila algoritma belum matang. Oleh karena itu, pembelajaran dan
pengujian akan menggunakan simulasi.

Sistem simulasi yang dibuat akan membutuhkan dua komputer:
\begin{enumerate}
	\item komputer untuk menjalankan simulasi (disebut server CARLA), dan
	\item komputer yang akan digunakan pada trem dan mengandung algoritma
	      kendali serta pembelajaran mesin (disebut server NVIDIA Pegasus).
\end{enumerate}
Kedua komputer (server) akan terhubung melalui sebuah jaringan menggunakan
sebuah protokol komunikasi. Sistem simulasi yang dibuat harus cukup cepat dalam
mengirimkan data antarkomputer, selain harus dapat mereplikasi keadaan dunia
nyata. Tugas akhir ini akan membahas pemilihan protokol komunikasi dan
pengembangan sistem simulasi tersebut.

Protokol komunikasi yang dipilih harus dapat mengirimkan data \textit{binary}.
Hal ini karena ada beberapa sensor yang disimulasikan dan data dari sensor
akan dikirim ke algoritma kendali dalam format tersebut. Selain \textit{binary},
protokol juga harus dapat mengirimkan data dalam format \textit{string} dan
\textit{integer}. Kedua data tersebut digunakan sebagai respon dari algoritma
kendali untuk mengendalikan trem.

Sistem simulasi yang dibuat bersifat HILS (\textit{hardware-in-the-loop
	simulation}). Artinya, perangkat keras yang digunakan pada simulasi juga akan
digunakan pada produk akhir trem otonom. Perangkat keras yang digunakan pada
produk akhir adalah server NVIDIA Pegasus.

\section{Rumusan Masalah}

Pada keadaan saat ini, server CARLA dan server NVIDIA Pegasus dihubungkan
menggunakan layanan web (\textit{web service}). Layanan web yang ada menggunakan
protokol HTTP untuk komunikasinya. Penggunaan HTTP untuk komunikasi pada layanan
web menyebabkan latensi yang besar ketika melakukan pertukuran data antarserver.
Karena adanya latensi besar ini, sistem simulasi menjadi tidak reliabel.

Berdasarkan latar belakang serta keadaan saat ini, definisi rumusan masalah
untuk tugas akhir ini adalah sebagai berikut.
\begin{enumerate}
	\item Bagaimana cara mengimplementasikan jalur komunikasi yang reliabel
	      untuk kebutuhan simulasi kendaraan otonom?
	\item Protokol atau kakas komunikasi apa yang paling tepat digunakan untuk
	      sistem simulasi kendaraan otonom?
\end{enumerate}

\section{Tujuan}

Tugas akhir ini memiliki 2 tujuan utama, yaitu
\begin{enumerate}
	\item mengimplementasikan jalur komunikasi yang reliabel untuk sistem
	      simulasi kendaraan otonom, dan
	\item menentukan protokol atau kakas komunikasi yang paling tepat untuk
	      sistem simulasi kendaraan otonom.
\end{enumerate}

\section{Batasan Masalah}

Batasan pada pengerjaan tugas akhr adalah

\begin{enumerate}
	\item pengujian akan dilakukan pada komputer yang terdapat di gedung Lab
	      Tek\-no\-lo\-gi VIII ITB; dan
	\item hanya ada 3 protokol atau kakas komunikasi yang diuji: ROS, ZeroMQ,
	      dan gRPC.
\end{enumerate}

\section{Metodologi}

% Tuliskan semua tahapan yang akan dilalui selama pelaksanaan tugas akhir.
% Tahapan ini spesifik untuk menyelesaikan persoalan tugas akhir. Tahapan studi
% literatur tidak perlu dituliskan karena ini adalah pekerjaan yang harus Anda
% lakukan selama proses pelaksanaan tugas akhir.

Tahapan pelaksanaan tugas akhir ini adalah sebagai berikut.

\begin{enumerate}
	\item Mempelajari CARLA.
	\item Mempelajari sistem saat ini.
	\item Eksplorasi kakas dan protokol yang akan digunakan.
	\item Perancangan kerangka kerja pengujian.
	\item Implementasi kerangka kerja pengujian.
	\item Eksperimen dan pengujian.
	\item Analisis hasil pengujian.
\end{enumerate}

\section{Jadwal Pelaksanaan Tugas Akhir}

% Tuliskan rencana kegiatan dan jadwal (dirinci sampai per minggu) mulai dari
% awal pelaksanaan Tugas Akhir I s.d. sidang tugas akhir berikut milestones dan
% deliverables yang harus diberikan. Jadwal ini dapat dibantu dengan membuat
% sebuah tabel timeline atau gantt chart.

Berikut adalah jadwal pelaksanaan Tugas Akhir I dan II yang dirinci per minggunya.

\begin{figure}[ht]
	\linespread{0.8}
	\resizebox{\textwidth}{!}{
		\begin{ganttchart}[
				y unit title=0.5cm,
				y unit chart=1.3cm,
				bar height=0.7,
				vgrid,
				title height=1,
				bar label font=\tiny,
				bar label node/.style={
						text width=4cm,
						align=right,
						anchor=east,
						font=\raggedleft
					},
			]{1}{20}
			%labels
			\gantttitle{September}{4}
			\gantttitle{Oktober}{4}
			\gantttitle{November}{4}
			\gantttitle{Desember}{4}
			\gantttitle{Januari}{4} \\
			\gantttitlelist{1,...,4}{1}
			\gantttitlelist{1,...,4}{1}
			\gantttitlelist{1,...,4}{1}
			\gantttitlelist{1,...,4}{1}
			\gantttitlelist{1,...,4}{1} \\

			% tasks
			\ganttbar{Mempelajari CARLA}{4}{4} \\
			\ganttbar{Mempelajari sistem saat ini}{5}{5} \\
			\ganttbar{Eksplorasi rosbridge}{6}{7} \\
			\ganttbar{Implementasi jalur komunikasi menggunakan
				rosbridge}{8}{9} \\
			\ganttbar{Mempelajari CARLA ROS Bridge}{10}{11} \\
			\ganttbar{Implementasi jalur komunikasi menggunakan ROS}{11}{13} \\
			\ganttbar{Eksplorasi protokol komunikasi lain}{14}{20} \\
		\end{ganttchart}
	}
	\caption{Jadwal Pelaksanaan Tugas Akhir (bagian 1)}
\end{figure}

\begin{figure}[ht]
	\linespread{0.8}
	\resizebox{\textwidth}{!}{
		\begin{ganttchart}[
				y unit title=0.5cm,
				y unit chart=1.3cm,
				bar height=0.7,
				vgrid,
				title height=1,
				bar label font=\tiny,
				bar label node/.style={
						text width=4cm,
						align=right,
						anchor=east,
						font=\raggedleft
					},
			]{1}{24}
			%labels
			\gantttitle{Februari}{4}
			\gantttitle{Maret}{4}
			\gantttitle{April}{4}
			\gantttitle{Mei}{4}
			\gantttitle{Juni}{4}
			\gantttitle{Juli}{4} \\
			\gantttitlelist{1,...,4}{1}
			\gantttitlelist{1,...,4}{1}
			\gantttitlelist{1,...,4}{1}
			\gantttitlelist{1,...,4}{1}
			\gantttitlelist{1,...,4}{1}
			\gantttitlelist{1,...,4}{1} \\

			% tasks
			\ganttbar{Perancangan kerangka kerja pengujian}{1}{8} \\
			\ganttbar{Implementasi kerangka kerja pengujian}{9}{16} \\
		\end{ganttchart}
	}
	\caption{Jadwal Pelaksanaan Tugas Akhir (bagian 2)}
\end{figure}