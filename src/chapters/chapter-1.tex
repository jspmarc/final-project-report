\chapter{Pendahuluan}

\section{Latar Belakang}

Pada tahun 2015, Perserikatan Bangsa-bangsa (PBB) mengeluarkan 17 SDG
(\textit{sustainable development goal}). Salah target yang ingin dicapai SGD
nomor 11 adalah menyediakan sistem transportasi yang aman, murah, mudah diakses,
dan berkelanjutan (\textit{sustainable}) untuk khalayak ramai. Untuk mendukung
tujuan tersebut, dikembangkan trem untuk digunakan sebagai transportasi publik
di Indonesia. Trem dipilih karena murah dan menggunakan listrik untuk bahan
bakarnya sehingga dapat menggunakan energi yang ramah lingkungan.

Trem diharapkan dapat menjadi alternatif bagi transportasi pribadi. Transportasi
pribadi semakin lama semakin tidak cocok untuk pembangunan berkelanjutan. Selain
alternatif, trem yang dikembangkan diharapkan juga dapat mengurangi masalah yang
ditimbulkan oleh kendaraan pribadi. Masalah-masalah tersebut termasuk, tapi
tidak terbatas pada, kemacetan dan emisi karbon dioksida.

Trem akan memanfaatkan teknologi inteligensi buatan dan pembelajaran mesin yang
memungkinkan kendaraan untuk beroperasi tanpa masinis secara otonom
(\textit{autopilot}). Dengan menggunakan teknologi ini, diharapkan trem otonom
dapat meminimalisasi kecelakaan karena galat yang disebabkan manusia sehingga
bisa menjadi pilihan transportasi yang aman.

Untuk mengembangkan teknologi pembelajaran mesin dan algoritma kendali
o\-to\-nom yang digunakan, perlu dikumpulkan data. Pengumpulan data ini akan
sangat mahal, butuh waktu yang lama, dan berbahaya bila dilakukan langsung di
lapangan. Oleh karena itu, data yang dikumpulkan akan dilakukan melalui
simulasi. Tugas akhir ini akan membahas pengembangan simulasi yang digunakan
untuk pengembangan trem otonom. Algoritma dan pembelajaran mesin juga akan
diujikan pada simulasi yang dibuat.

Sistem simulasi yang dibuat akan membutuhkan dua komputer:
\begin{enumerate}
	\item komputer untuk menjalankan simulasi (disebut server CARLA), dan
	\item komputer yang akan digunakan pada trem dan mengandung algoritma
	      kendali serta pembelajaran mesin (disebut server NVIDIA Pegasus).
\end{enumerate}
Sistem simulasi yang dibuat, selain harus dapat mereplikasi keadaan dunia nyata,
harus cukup cepat dalam memproses dan mengirimkan data antarkomputer. Hal metode
pengiriman data antara kedua komputer tersebut akan dibahas pada tugas akhir
ini.

Server NVIDIA Pegasus yang digunakan pada sistem simulasi ini juga akan
digunakan pada produk akhir trem otonom. Artinya, simulasi yang dilakukan
berbentuk HILS (\textit{hardware-in-the-loop simulation}).

\section{Rumusan Masalah}

Pada keadaan saat ini, server CARLA dan server NVIDIA Pegasus dihubungkan
menggunakan layanan web (\textit{web service}). Layanan web yang ada menggunakan
protokol HTTP untuk komunikasinya. Penggunaan HTTP untuk komunikasi pada layanan
web menyebabkan latensi yang besar ketika melakukan pertukuran data
antarserver.

Karena adanyayy latensi besar ini, sistem simulasi menjadi tidak reliabel.
Selain itu, proses pengembangan algoritma dan inteligensi buatan juga terhambat
karena sistem simulasi menjadi lebih lambat dari seharusnya. Perlu ditemukan
metode lain yang dapat menghilangkan \textit{bottleneck} pada jalur
komunikasi sehingga sistem simulasi dapat digunakan.

Untuk memecahkan masalah latensi tersebut, ada beberapa solusi yang akan dibahas
pada tugas akhir ini. Alternatif-alternatif tersebut adalah ROS, ZeroMQ, dan
Apache Thrift. komunikasi pada suatu sistem terdistribusi yang lebih cepat
daripada HTTP. ROS dipilih karena kerap kali dipilih pada sistem simulasi HILS.

\section{Tujuan}

\begin{itemize}
	\item Berisi tujuan utama yang dihasilkan
	\item Dapat ditambah dengan tujuan detil yang akan dicapai dalam pelaksanaan
	      tugas akhir.
	\item Fokuskan pada hasil akhir yang ingin diperoleh setelah tugas akhir
	      diselesaikan, terkait dengan penyelesaian persoalan yang telah disampaikan
	      pada rumusan masalah.
\end{itemize}

\section{Batasan Masalah}

Tuliskan batasan-batasan yang diambil dalam pelaksanaan tugas akhir (batasan
kajian, bkn batasan implementasi). Batasan ini dapat dihindari (tidak perlu ada)
jika topik/judul tugas akhir dibuat cukup spesifik.

\section{Metodologi}

Tuliskan semua tahapan yang akan dilalui selama pelaksanaan tugas akhir.
Tahapan ini spesifik untuk menyelesaikan persoalan tugas akhir. Tahapan studi
literatur tidak perlu dituliskan karena ini adalah pekerjaan yang harus Anda
lakukan selama proses pelaksanaan tugas akhir.

\section{Jadwal Pelaksanaan Tugas Akhir}

Tuliskan rencana kegiatan dan jadwal (dirinci sampai per minggu) mulai dari
awal pelaksanaan Tugas Akhir I s.d. sidang tugas akhir berikut milestones dan
deliverables yang harus diberikan. Jadwal ini dapat dibantu dengan membuat
sebuah tabel timeline atau gantt chart.
