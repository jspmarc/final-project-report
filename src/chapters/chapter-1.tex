\chapter{Pendahuluan}

\section{Latar Belakang}

Pada tahun 2015, Perserikatan Bangsa-bangsa (PBB) mengeluarkan 17 SDG
(\textit{sustainable development goal}). Salah target yang ingin dicapai SGD
nomor 11 adalah menyediakan sistem transportasi yang aman, murah, mudah diakses,
dan berkelanjutan (\textit{sustainable}) untuk khalayak ramai. Untuk mendukung
tujuan tersebut, dikembangkan trem untuk digunakan sebagai transportasi publik
di Indonesia. Trem dipilih karena murah dan menggunakan listrik untuk bahan
bakarnya sehingga dapat menggunakan energi yang ramah lingkungan.

Trem diharapkan dapat menjadi alternatif bagi transportasi pribadi. Transportasi
pribadi semakin lama semakin tidak cocok untuk pembangunan berkelanjutan. Selain
alternatif, trem yang dikembangkan diharapkan juga dapat mengurangi masalah yang
ditimbulkan oleh kendaraan pribadi. Masalah-masalah tersebut termasuk, tapi
tidak terbatas pada, kemacetan dan emisi karbon dioksida.

Trem akan memanfaatkan teknologi inteligensi buatan dan pembelajaran mesin yang
memungkinkan kendaraan untuk beroperasi tanpa masinis secara otonom
(\textit{autopilot}). Dengan menggunakan teknologi ini, diharapkan trem otonom
dapat meminimalisasi kecelakaan karena galat yang disebabkan manusia sehingga
bisa menjadi pilihan transportasi yang aman.

Untuk mengembangkan teknologi pembelajaran mesin dan algoritma kendali otonom
yang digunakan, perlu dikumpulkan data. Pengumpulan data ini akan sangat mahal,
butuh waktu yang lama, dan berbahaya bila dilakukan langsung di lapangan. Oleh
karena itu, data yang dikumpulkan akan dilakukan melalui simulasi. Tugas akhir
ini akan membahas pengembangan simulasi yang digunakan untuk pengembangan trem
otonom. Algoritma dan pembelajaran mesin juga akan diujikan pada simulasi yang
dibuat.

Sistem simulasi yang dibuat akan membutuhkan dua komputer:
\begin{enumerate}
	\item komputer untuk menjalankan simulasi (disebut server CARLA), dan
	\item komputer yang akan digunakan pada trem dan mengandung algoritma
	      kendali serta pembelajaran mesin (disebut server NVIDIA Pegasus).
\end{enumerate}
Sistem simulasi yang dibuat, selain harus dapat mereplikasi keadaan dunia nyata,
harus cukup cepat dalam memproses dan mengirimkan data antar-komputer. Hal
metode pengiriman data antara kedua komputer tersebut akan dibahas pada tugas
akhir ini.

\section{Rumusan Masalah}

Rumusan Masalah berisi masalah utama yang dibahas dalam tugas akhir. Rumusan
masalah yang baik memiliki struktur sebagai berikut:

\begin{enumerate}
	\item Penjelasan ringkas tentang kondisi/situasi yang ada sekarang terkait
	      dengan topik utama yang dibahas Tugas Akhir.
	\item Pokok persoalan dari kondisi/situasi yang ada, dapat dilihat dari
	      kelemahan atau kekurangannya. \textbf{Bagian ini merupakan inti dari rumusan
		      masalah}.
	\item Elaborasi lebih lanjut yang menekankan pentingnya untuk menyelesaikan
	      pokok persoalan tersebut.
	\item Usulan singkat terkait dengan solusi yang ditawarkan untuk
	      menyelesaikan persoalan.
\end{enumerate}

\section{Tujuan}

\begin{itemize}
	\item Berisi tujuan utama yang dihasilkan
	\item Dapat ditambah dengan tujuan detil yang akan dicapai dalam pelaksanaan
	      tugas akhir.
	\item Fokuskan pada hasil akhir yang ingin diperoleh setelah tugas akhir
	      diselesaikan, terkait dengan penyelesaian persoalan yang telah disampaikan
	      pada rumusan masalah.
\end{itemize}

\section{Batasan Masalah}

Tuliskan batasan-batasan yang diambil dalam pelaksanaan tugas akhir (batasan
kajian, bkn batasan implementasi). Batasan ini dapat dihindari (tidak perlu ada)
jika topik/judul tugas akhir dibuat cukup spesifik.

\section{Metodologi}

Tuliskan semua tahapan yang akan dilalui selama pelaksanaan tugas akhir.
Tahapan ini spesifik untuk menyelesaikan persoalan tugas akhir. Tahapan studi
literatur tidak perlu dituliskan karena ini adalah pekerjaan yang harus Anda
lakukan selama proses pelaksanaan tugas akhir.

\section{Jadwal Pelaksanaan Tugas Akhir}

Tuliskan rencana kegiatan dan jadwal (dirinci sampai per minggu) mulai dari
awal pelaksanaan Tugas Akhir I s.d. sidang tugas akhir berikut milestones dan
deliverables yang harus diberikan. Jadwal ini dapat dibantu dengan membuat
sebuah tabel timeline atau gantt chart.
