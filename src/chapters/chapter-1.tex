\chapter{Pendahuluan}

\section{Latar Belakang}

Pada tahun 2015, Perserikatan Bangsa-bangsa (PBB) mengeluarkan 17 SDG
(\textit{sustainable development goal}). Salah target yang ingin dicapai SGD
nomor 11 adalah menyediakan sistem transportasi yang aman, murah, mudah diakses,
dan berkelanjutan (\textit{sustainable}) untuk khalayak ramai. Untuk mendukung
tujuan tersebut, dikembangkan trem untuk digunakan sebagai transportasi publik
di Indonesia. Trem dipilih karena murah dan menggunakan listrik untuk bahan
bakarnya sehingga dapat menggunakan energi yang ramah lingkungan.

Trem diharapkan dapat menjadi alternatif bagi transportasi pribadi. Transportasi
pribadi semakin lama semakin tidak cocok untuk pembangunan berkelanjutan. Selain
alternatif, trem yang dikembangkan diharapkan juga dapat mengurangi masalah yang
ditimbulkan oleh kendaraan pribadi. Masalah-masalah tersebut termasuk, tapi
tidak terbatas pada, kemacetan dan emisi karbon dioksida.

Trem akan memanfaatkan teknologi inteligensi buatan dan pembelajaran mesin yang
memungkinkan kendaraan untuk beroperasi tanpa masinis secara otonom
(\textit{autopilot}). Dengan menggunakan teknologi ini, diharapkan trem otonom
dapat meminimalisasi kecelakaan karena galat yang disebabkan manusia sehingga
bisa menjadi pilihan transportasi yang aman.

Untuk mengembangkan teknologi pembelajaran mesin dan algoritma kendali
o\-to\-nom yang digunakan, perlu dikumpulkan data. Pengumpulan data ini akan
sangat mahal, butuh waktu yang lama, dan berbahaya bila dilakukan langsung di
lapangan. Oleh karena itu, data yang dikumpulkan akan dilakukan melalui
simulasi. Tugas akhir ini akan membahas pengembangan simulasi yang digunakan
untuk pengembangan trem otonom. Algoritma dan pembelajaran mesin juga akan
diujikan pada simulasi yang dibuat.

Sistem simulasi yang dibuat akan membutuhkan dua komputer:
\begin{enumerate}
	\item komputer untuk menjalankan simulasi (disebut server CARLA), dan
	\item komputer yang akan digunakan pada trem dan mengandung algoritma
	      kendali serta pembelajaran mesin (disebut server NVIDIA Pegasus).
\end{enumerate}
Sistem simulasi yang dibuat, selain harus dapat mereplikasi keadaan dunia nyata,
harus cukup cepat dalam memproses dan mengirimkan data antarkomputer. Hal metode
pengiriman data antara kedua komputer tersebut akan dibahas pada tugas akhir
ini.

Server NVIDIA Pegasus yang digunakan pada sistem simulasi ini juga akan
digunakan pada produk akhir trem otonom. Artinya, simulasi yang dilakukan
berbentuk HILS (\textit{hardware-in-the-loop simulation}).

\section{Rumusan Masalah}

Pada keadaan saat ini, server CARLA dan server NVIDIA Pegasus dihubungkan
menggunakan layanan web (\textit{web service}). Layanan web yang ada menggunakan
protokol HTTP untuk komunikasinya. Penggunaan HTTP untuk komunikasi pada layanan
web menyebabkan latensi yang besar ketika melakukan pertukuran data
antarserver. Karena adanya latensi besar ini, sistem simulasi menjadi tidak reliabel.

Berdasarkan latar belakang serta keadaan saat ini, definisi rumusan masalah
untuk tugas akhir ini adalah sebagai berikut.
\begin{enumerate}
	\item Bagaimana cara mengimplementasikan jalur komunikasi yang reliabel
	      untuk kebutuhan simulasi kendaraan otonom?
	\item Protokol atau kakas komunikasi apa yang paling tepat digunakan untuk
	      sistem simulasi kendaraan otonom?
\end{enumerate}

\section{Tujuan}

Tugas akhir ini memiliki 2 tujuan utama, yaitu
\begin{enumerate}
	\item mengimplementasikan jalur komunikasi yang reliabel untuk sistem
	      simulasi kendaraan otonom, dan
	\item menentukan protokol atau kakas komunikasi yang paling tepat untuk
	      sistem simulasi kendaraan otonom.
\end{enumerate}

\section{Batasan Masalah}

Batasan pada pengerjaan tugas akhr adalah

\begin{enumerate}
	\item pengujian akan dilakukan pada komputer yang terdapat di gedung Lab
	      Tek\-no\-lo\-gi VIII ITB;
	\item hanya ada 3 protokol atau kakas komunikasi yang diuji: ROS, ZeroMQ,
	      dan Apache Thrift; dan
	\item jalur komunikasi hanya akan dinilai berdasarkan latensi.
\end{enumerate}

\section{Metodologi}

Tuliskan semua tahapan yang akan dilalui selama pelaksanaan tugas akhir.
Tahapan ini spesifik untuk menyelesaikan persoalan tugas akhir. Tahapan studi
literatur tidak perlu dituliskan karena ini adalah pekerjaan yang harus Anda
lakukan selama proses pelaksanaan tugas akhir.

\section{Jadwal Pelaksanaan Tugas Akhir}

Tuliskan rencana kegiatan dan jadwal (dirinci sampai per minggu) mulai dari
awal pelaksanaan Tugas Akhir I s.d. sidang tugas akhir berikut milestones dan
deliverables yang harus diberikan. Jadwal ini dapat dibantu dengan membuat
sebuah tabel timeline atau gantt chart.
