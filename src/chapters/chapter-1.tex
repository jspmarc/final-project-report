\chapter{Pendahuluan}

\section{Latar Belakang}

Dalam pengembangan kota-kota di Indonesia, dibutuhkan kendaraan umum yang aman,
murah, mudah diakses, dan berkelanjutan (\textit{sustainable}). Salah satu moda
transportasi umum yang dapat memenuhi keempat hal tersebut adalah trem listrik.
Agar bisa lebih aman, trem listrik akan memanfaatkan inteligensi buatan dan
pembelajaran mesin sehingga trem tidak membutuhkan masinis dan bisa beroperasi
secara otonom.

Pengembangan teknologi pembelajaran mesin dan algoritma kendali otonom yang
digunakan memerlukan banyak data dan pengujian. Proses pengumpulan data dan
pengujian ini akan mahal dan membutuhkan waktu yang lama. Selain itu, akan
berbahaya apabila algoritma belum matang. Oleh karena itu, pembelajaran dan
pengujian akan menggunakan simulasi.

Sistem simulasi yang dibuat akan membutuhkan dua komputer:
\begin{enumerate}
	\item komputer untuk menjalankan simulasi (disebut server CARLA), dan
	\item komputer yang akan digunakan pada trem dan mengandung algoritma
	      kendali serta pembelajaran mesin (disebut server NVIDIA Pegasus).
\end{enumerate}
Kedua komputer (server) akan terhubung melalui sebuah jaringan menggunakan
sebuah protokol komunikasi. Sistem simulasi yang dibuat harus cukup cepat dalam
mengirimkan data antarkomputer, selain harus dapat mereplikasi keadaan dunia
nyata. Tugas akhir ini akan membahas pemilihan protokol komunikasi dan
pengembangan sistem simulasi tersebut.

Protokol komunikasi yang dipilih harus dapat mengirimkan data \textit{binary}.
Hal ini karena ada beberapa sensor yang disimulasikan dan data dari sensor
akan dikirim ke algoritma kendali dalam format tersebut. Selain \textit{binary},
protokol juga harus dapat mengirimkan data dalam format \textit{string} dan
\textit{integer}. Kedua data tersebut digunakan sebagai respon dari algoritma
kendali untuk mengendalikan trem.

Sistem simulasi yang dibuat bersifat HILS (\textit{hardware-in-the-loop
	simulation}). Artinya, perangkat keras yang digunakan pada simulasi juga akan
digunakan pada produk akhir trem otonom. Perangkat keras yang digunakan pada
produk akhir adalah server NVIDIA Pegasus.

\section{Rumusan Masalah}

Pada keadaan saat ini, server CARLA dan server NVIDIA Pegasus dihubungkan
menggunakan layanan web (\textit{web service}). Layanan web yang ada menggunakan
protokol HTTP untuk komunikasinya. Penggunaan HTTP untuk komunikasi pada layanan
web menyebabkan latensi yang besar ketika melakukan pertukuran data antarserver.
Karena adanya latensi besar ini, sistem simulasi menjadi tidak reliabel.

Berdasarkan latar belakang serta keadaan saat ini, definisi rumusan masalah
untuk tugas akhir ini adalah sebagai berikut.
\begin{enumerate}
	\item Bagaimana cara mengimplementasikan jalur komunikasi yang reliabel
	      untuk kebutuhan simulasi kendaraan otonom?
	\item Protokol atau kakas komunikasi apa yang paling tepat digunakan untuk
	      sistem simulasi kendaraan otonom?
\end{enumerate}

\section{Tujuan}

Tugas akhir ini memiliki 2 tujuan utama, yaitu
\begin{enumerate}
	\item mengimplementasikan jalur komunikasi yang reliabel untuk sistem
	      simulasi kendaraan otonom, dan
	\item menentukan protokol atau kakas komunikasi yang paling tepat untuk
	      sistem simulasi kendaraan otonom.
\end{enumerate}

\section{Batasan Masalah}

Batasan pada pengerjaan tugas akhr adalah

\begin{enumerate}
	\item pengujian akan dilakukan pada komputer yang terdapat di gedung Lab
	      Tek\-no\-lo\-gi VIII ITB; dan
	\item hanya ada 3 protokol atau kakas komunikasi yang diuji: ROS, ZeroMQ,
	      dan gRPC.
\end{enumerate}

\section{Metodologi}

% Tuliskan semua tahapan yang akan dilalui selama pelaksanaan tugas akhir.
% Tahapan ini spesifik untuk menyelesaikan persoalan tugas akhir. Tahapan studi
% literatur tidak perlu dituliskan karena ini adalah pekerjaan yang harus Anda
% lakukan selama proses pelaksanaan tugas akhir.

Tahapan pelaksanaan tugas akhir ini adalah sebagai berikut.

\begin{enumerate}
	\item Mempelajari CARLA.
	\item Mempelajari sistem saat ini.
	\item Eksplorasi kakas dan protokol yang akan digunakan.
	\item Perancangan kerangka kerja pengujian.
	\item Implementasi kerangka kerja pengujian.
	\item Eksperimen dan pengujian.
	\item Analisis hasil pengujian.
\end{enumerate}

\section{Sistematika Pembahasan}

% Subbab ini berisi penjelasan ringkas isi per bab. Penjelasan ditulis satu
% paragraf per bab buku.

Pada buku ini akan terdapat 5 bab, yaitu
\begin{enumerate}
	\item Pendahuluan,
	\item Studi Literatur,
	\item Analisis dan Rancangan Metode Komunikasi Sistem HILS,
	\item Implementasi dan Pengujian Metode Komunikasi Sistem HILS, dan
	\item Kesimpulan dan Saran.
\end{enumerate}

Bab pertama adalah bab pendahuluan yang membahas masalah dan latar belakang
masalah. Lalu, dari masalah tersebut dirumuskan tujuan dari penelitian yang
dilakukan. Agar penelitian tidak terlalu luas, bab pendahuluan juga membahas
batasan masalah. Terakhir, bab pendahuluan juga membahas metodologi penelitian.

Selanjutnya adalah bab studi literatur. Sesuai namanya, bab ini membahas hasil
studi literatur terkait topik TA yang diangkat, yaitu membuat sistem HILS untuk
simulasi trem \textit{autonomous}. Bab ini akan menjabarkan perangkat lunak, yang
digunakan, misalnya \textit{simulator} CARLA dan ZeroMQ, serta hasil penelitian
sebelumnya yang terkait topik tugas akhir.

Bab ketiga adalah analisis dan rancangan metode komunikasi untuk sistem HILS.
Pada bab ini akan membahas hasil analisis sistem dalam bentuk kebutuhan
non-fungsional sistem. Lalu, dari hasil analisis tersebut akan dibuat rancangan
untuk sistem HILS serta metode komunikasinya.

Setelah dianalisis dan dirancang, dilakukan implementasi dan pengujian. Hasil
implementasi dan pengujian dituangkan pada bab keempat buku ini.

Setelah semuanya selesai, dituliskan kesimpulan dan saran terkait proses
pembuatan tugas akhir dan penulisan buku ini.