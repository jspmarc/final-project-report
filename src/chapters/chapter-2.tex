\chapter{Studi Literatur}

% Bab Studi Literatur digunakan untuk mendeskripsikan kajian literatur yang
% terkait dengan persoalan tugas akhir. Tujuan studi literatur adalah:

% \begin{enumerate}
%     \item menunjukkan kepada pembaca adanya gap seperti pada rumusan masalah
%         yang memang belum terselesaikan,
%     \item memberikan pemahaman yang secukupnya kepada pembaca tentang teori
%         atau pekerjaan terkait yang terkait langsung dengan penyelesaian
%         persoalan, serta
%     \item menyampaikan informasi apa saja yang sudah ditulis/dilaporkan oleh
%         pihak lain (peneliti/Tugas Akhir/Tesis) tentang hasil
%         penelitian/pekerjaan mereka yang sama atau mirip kaitannya dengan
%         persoalan tugas akhir.
% \end{enumerate}
Pada bab ini, Penulis akan menguraikan hasil literatur dalam penyusunan tugas
akhir ini. Subbab pertama membahas simulator CARLA, yaitu perangkat lunak yang
digunakan sebagai alat simulasi. Subbab kedua menjelaskan NVIDIA Pegasus yang
akan digunakan sebagai mesin untuk menjalankan algoritma \textit{decision
making} di lingkungan simulasi dan \textit{production}. Lalu, pada subbab
ketiga akan dibahas beberapa cara komunikasi yang dapat digunakan pada sistem
terdistribusi. Terakhir, subbab keempat akan membahas penelitian-penelitian
terkait simulasi \textit{autonomous vehicle} menggunakan CARLA.

% Sekarang mau ke bab berapa yaaaa.... hmm... ke bab \ref{sec:latarbelakang} ahhhhh. 

\section{Simulator CARLA}
\blindtext

\section{NVIDIA Pegasus}

NVIDIA Pegasus adalah salah satu produk cetusan NVIDIA Corporation di bawah
lini produk NVIDIA Drive PX. Nama pasar dari NVIDIA Pegasus adalah NVIDIA Drive
PX Pegasus. Lini produk NVIDIA Drive sendiri merupakan platform komputer untuk
memberikan fungsionalitas bantuan mengemudi pada kendaraan bermotor.

Mengutip dari artikel \parencite{oh_2017}, NVIDIA Pegasus adalah komputer yang
mendukung pengemudian \textit{autonomous} secara penuh. Artinya, NVIDIA Pegasus
dapat digunakan untuk membuat sebuah kendaraan bermotor menjadi
\textit{autonomous vehcile} jika sensor dan algoritma yang digunakan tepat.

NVIDIA Pegasus menggunakan 2 GPU dengan arsitektur post-Volta dan 2 SoC NVIDIA
Xavier. Kombinasi CPU dan GPU ini dapat menghasilkan 320 TOPS (\textit{trillion
operations/second}) untuk komputasi intelegensi buatan. Untuk koneksi I/O,
NVIDIA Pegasus mendukung sampai dengan 16 kamera (6 di antaranya adalah lidar).

\section{Jalur Komunikasi pada Sistem Terdistribusi}
\blindtext

\section{Penelitian Terkait}
\blindtext

