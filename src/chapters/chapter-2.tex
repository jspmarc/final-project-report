\chapter{Studi Literatur}\label{chapter-2}

Pada Bab \ref{chapter-2}, Penulis akan menguraikan hasil studi literatur dalam
penyusunan tugas akhir ini. Subbab \ref{chapter-2-section-trem} membahas trem,
yaitu kendaraan yang akan disimulasikan menjadi otonom pada tugas akhir ini.
Selanjutnya, pada Subbab \ref{chapter-2-section-carla} akan dibahas simulator
CARLA, yaitu perangkat lunak yang digunakan sebagai alat simulasi.  Subbab
\ref{chapter-2-section-pegasus} menjelaskan NVIDIA Pegasus yang akan digunakan
sebagai mesin untuk menjalankan algoritma \textit{decision making} di lingkungan
simulasi dan \textit{production}. Lalu, Subbab
\ref{chapter-2-section-driveworks} akan membahas NVIDIA DriveWorks sebagai SDK
(\textit{software development kit}) pengembangan perangkat lunak pada NVIDIA
Pegasus. Lalu, pada subbab \ref{chapter-2-section-comms} akan dibahas kondisi
metode komunikasi jalur komunikasi serta ZeroMQ dan ROS2 sebagai alternatif ke
kondisi yang ada. Terakhir, subbab \ref{chapter-2-section-related-works} akan
membahas sebuah penelitian terkait simulasi \textit{autonomous vehicle}
menggunakan CARLA dengan HILS.

\section{Trem}\label{chapter-2-section-trem}

Trem adalah transportasi umum berupa kereta ringan yang memiliki rel/jalur
sendiri di jalan raya. Artinya trem harus berbagi dengan pengguna jalan raya
lain, seperti mobil, sepeda motor, dan pejalan kaki. Meskipun demikian, trem
dapat menjadi solusi terhadap masalah yang semakin besar di kota-kota besar,
yaitu kemacetan dan masalah lingkungan \parencite{trilaksono_laporanRispro}.

Trem dapat mengatasi kemacetan karena trem dapat memuat hingga 125--250
penumpang.  Akan tetapi, trem harus berjalan lebih lambat untuk menghindari
kecelakaan dengan pengguna jalan raya lain \parencite{trilaksono_laporanRispro}.
Lalu, kebanyakan trem menggunakan listrik sehingga tidak ada emisi karbon
dioksida di kota dan tidak banyak mengeluarkan suara.

Di Indonesia sendiri, trem sudah ada sejak jaman penjajahan. Trem banyak
digunakan di Ibu Kota Negara, Jakarta. Akan tetapi, trem di Jakarta ditutup pada
tahun 1960-an \parencite{adryamarthanino_sejarahTremDiJakarta}. Untungnya,
dengan berkembangnya kecemasan untuk masalah lalu lintas, polusi, dan
pertumbuhan kota yang pesat, mulai banyak kota yang ingin mengadopsi trem
kembali. Pemerintah Kota Bogor, Provinsi Bali, dan Kota Surabaya sudah tertarik
lagi untuk menggunakan trem sebagai transportasi perkotaan
\parencite{trilaksono_laporanRispro}.

\section{Simulator CARLA}\label{chapter-2-section-carla}

CARLA (\textit{Car Learning to Act}) adalah perangkat lunak sumber terbuka
(\textit{open source}) yang memiliki tujuan utama menyimulasikan kendaraan
\textit{autonomous}. CARLA dibangun dari nol untuk mampu melakukan pelatihan,
pembuatan prototipe, dan validasi model kemudi otonom, termasuk persepsi dan
kontrol dari model tersebut \parencite{dos_carla}. Antarmuka CARLA dapat dilihat
pada Gambar \ref{chapter-2-fig-carla-ui}.

\begin{figure}[!htbp]
	\centering
	\includegraphics[width=0.5\textwidth]{resources/chapter-2/CARLA-cropped.png}
	\caption{Antarmuka simulator CARLA}
	\label{chapter-2-fig-carla-ui}
\end{figure}

CARLA sendiri dibangun di atas mesin gim (\textit{game engine}) Unreal Engine 4
(UE4). UE4 dipilih karena memiliki simulasi fisika dan kualitas grafis terbaik,
setidaknya pada saat CARLA dibuat. Selain itu, UE4 juga memiliki ekosistem untuk
pengaya (\textit{plugin}) dan sistem untuk menambahkan logika dasar untuk NPC
(\textit{non-playable character}\footnote{Istilah untuk karakter yang tidak
	dapat dimainkan pada gim.}). UE4 dan sifat sumber terbuka CARLA memungkinkan
komunitas untuk menambahkan berbagai hal sesuai dengan kebutuhan mereka
\parencite{dos_carla} atau bahkan memperbaiki CARLA.

Agar simulasi yang dilakukan CARLA serealistis mungkin, CARLA menyediakan
berbagai macam model 3D: kendaraan, pejalan kaki, gedung, dan rambu lalu lintas.
Model-model 3D yang disediakan dibuat dengan menggunakan teknik yang
menghasilkan bentuk dan tekstur senyata mungkin, akan tetapi tetap cepat untuk
di-\textit{render} \parencite{dos_carla}. Model kendaraan dan pejalan kaki yang
dibuat bisa memiliki banyak variasi, misalnya kendaraan Yamaha YZF dengan warna
yang berbeda atau pejalan kaki dengan aksesoris dan pakaian yang berbeda.
Model-model 3D digunakan untuk membangun lingkungan simulasi yang realistis.

Selain dari model 3D, lingkungan simulasi yang realistis didapatkan juga dari
cuaca dan waktu hari. Cuaca dan waktu hari pada CARLA dapat dikustomisasi
sedemikian rupa untuk memungkinkan berbagai macam skenario simulasi.
Selain itu, NPC yang ada di CARLA akan menggunakan sebuah model dari kendaraan
ataupun pejalan kaki. NPC bergerak berdasarkan suatu aturan yang sudah
ditentukan, misalnya mengikuti suatu pemilihan keputusan ketika di lampu lalu
lintas \parencite{dos_carla}.

Pada CARLA, model-model tersebut di-\textit{package} ke dalam CARLA dalam bentuk
\textit{blueprint}. \textit{Blueprint} adalah model yang sudah dibuat dengan
animasi dan beberapa atribut. \textit{Blueprint} dapat di-\textit{spawn} menjadi
aktor CARLA. Aktor dan \textit{blueprint} pada CARLA tidak terbatas hanya pada
model kendaraan, NPC, dll., tetapi juga dapat berupa sensor dan rambu jalan.
Aktor kendaraan dapat juga dapat digunakan untuk membuat agen pada CARLA. Agen
adalah \textit{script} yang dapat digunakan untuk mengendalikan kendaraan. Cara
kerja agen dapat diubah dengan mengubah \textit{behavior} agen
\parencite{x_carlaDocs}.

Selain agen dan \textit{blueprint}, di CARLA juga terdapat istilah \textit{ego
	vehicle}. \textit{Ego vehicle} adalah agen yang digunakan untuk mengambil
data dunia CARLA dan dikendalikan pengguna CARLA. Sensor dapat dipasangkan
ke \textit{ego vehicle}. Karena \textit{ego vehicle} adalah agen, \textit{ego
	vehicle} di-\textit{spawn} dengan menggunakan \textit{blueprint} kendaraan
\parencite{x_carlaDocs}. \textit{Ego vehicle} dapat dikendalikan dengan
memberikan kendali seperti gas, rem, dan belokan setir.

Selain data dari sensor, data keadaan kendaraan juga bisa didapatkan dari CARLA
\textit{measurement}.  Data CARLA \textit{measurement} ini mengandung data id
agen, posisi x dan y agen, kecepatan, dan orientasi agen
\parencite{trilaksono_laporanRispro}.

CARLA bekerja dengan arsitektur server-klien. Arsitektur ini memungkinkan dunia
yang dinamis dan antarmuka yang sederhana antara dunia dan agen yang
berinteraksi dengannya. Server pada CARLA berfungsi untuk menjalankan simulasi
dan me-\textit{render scene}-nya. Lalu, klien bertugas untuk mengirimkan
perintah dan ``perintah meta'' ke server dan menerima data sensor. Perintah yang
dapat dikirimkan klien digunakan untuk mengendalikan kendaraan yang
disimulasikan. Sedangkan ``perintah meta'' digunakan untuk mengatur lingkungan
simulasi, misalnya cuaca, sensor yang digunakan, dan jumlah kendaraan/pejalan
kaki NPC. Klien CARLA berbentuk API yang ditulis dalam bahasa Python
\parencite{dos_carla}.

\section{NVIDIA Pegasus}\label{chapter-2-section-pegasus}

NVIDIA Pegasus adalah salah satu produk cetusan NVIDIA Corporation di bawah lini
produk NVIDIA Drive PX. Nama pasar dari NVIDIA Pegasus adalah NVIDIA Drive
PX Pegasus. Lini produk NVIDIA Drive sendiri merupakan platform komputer untuk
memberikan fungsionalitas bantuan mengemudi pada kendaraan bermotor
\parencite{oh_2017}.

% \begin{figure}[!htbp]
% 	\centering
% 	\includegraphics[width=\textwidth,trim={2cm 0cm 0cm 6cm},clip]{resources/chapter-2/pegasus.png}
% 	\caption{NVIDIA Pegasus \parencite{trilaksono_laporanRispro}}
% 	\label{chapter-2-fig-pegasus}
% \end{figure}

NVIDIA Pegasus adalah komputer yang mendukung pengemudian \textit{autonomous}
secara penuh \parencite{oh_2017}. Artinya, NVIDIA Pegasus dapat digunakan untuk
membuat sebuah kendaraan bermotor menjadi \textit{autonomous vehicle} jika
sensor dan algoritma yang digunakan tepat.

NVIDIA Pegasus menggunakan 2 GPU dengan arsitektur post-Volta dan 2 SoC NVIDIA
Xavier. Kombinasi CPU dan GPU ini dapat menghasilkan 320 TOPS (\textit{trillion
	operations/second}) untuk komputasi intelegensi buatan. Untuk koneksi I/O,
NVIDIA Pegasus mendukung sampai dengan 16 sensor dan 6 di antaranya adalah
kamera \parencite{oh_2017}.

\section{NVIDIA DriveWorks}\label{chapter-2-section-driveworks}

NVIDIA DriveWorks adalah SDK (\textit{software development kit}) untuk
mengembangkan perangkat lunak yang digunakan pada kendaraan otonom dengan
perangkat keras dari NVIDIA AGX. NVIDIA DriveWorks menyediakan berbagai kakas yang
dapat  digunakan untuk mempercepat pengembangan algoritma dan perangkat lunak
untuk kendaraan otonom. Kemampuan dari NVIDIA DriveWorks juga dapat dikembangkan
lagi oleh pemrogram karena sifatnya yang modular, terbuka, dan
\textit{customizable} \parencite{nvidia_driveworksMainSite}.

NVIDIA DriveWorks sendiri sudah mencapai versi 5. Akan tetapi, pada tugas
akhir ini akan digunakan NVIDIA DriveWorks 4. NVIDIA DriveWorks 4 dapat
dijalankan pada perangkat keras  NVIDIA Pegasus dan juga pada GPU
\textit{desktop} NVIDIA dengan \textit{chip} Turing atau Volta, misalnya NVIDIA
RTX 20-\textit{series} \parencite{nvidia_driveworksSdkGettingStarted}.

Beberapa pustaka dan kakas yang disediakan NVIDIA DriveWorks adalah visualisasi
(GUI), pengoptimal TensorRT, lapisan abstraksi kendaraan, perekam data, dan
\textit{logging} dan diagnosis sistem. Selain itu, beberapa fitur utama dari
NVIDIA DriveWorks adalah
\begin{enumerate}
	\item lapisan abstraksi sensor: memberikan antarmuka yang sama untuk
	      memanipulasi semua sensor yang didukung, kemampuan untuk membuat sensor
	      virtual menggunakan data yang direkam, dan lain-lain;
	\item pemroses gambar dan \textit{point cloud} (data lidar): terdapat
	      berbagai pemroses gambar seperti \textit{detection} dan
	      \textit{tracking} dan pemroses \textit{point cloud} seperti segmentasi
	      planar dengan abstraksi perangkat keras sehingga beberapa fitur
	      pemrosesan dapat dipercepat pada NVIDIA Drive tertentu; dan
	\item kerangka kerja DNN (\textit{deep neural network}): kerangka kerja yang
	      dapat digunakan untuk mamuat dan memprediksi menggunakan model ML
	      TensorRT yang sudah di-\textit{train}
	      \parencite{nvidia_driveworksSdkGettingStarted}.
\end{enumerate}

\section{Metode Komunikasi antara Simulator CARLA dan NVIDIA
  Pegasus}\label{chapter-2-section-comms}

Pada keadaan yang ada, komunikasi antara komputer SILS dengan komputer RKB/AGX
menggunakan perantara \textit{web service} yang berbasis HTTP (dapat dilihat
pada Gambar \ref{chapter-2-old-hils}). Penggunaan HTTP pada \textit{web service}
sendiri sebenarnya bukan \textit{bottleneck}/penghambat kinerja terbesar sistem
simulasi. \textit{Bottleneck} terbesar adalah operasi I/O yang dilakukan setiap
kali dilakukan pengiriman data. Dilakukan delapan operasi I/O yang berjalan
secara sinkronis untuk mengirimkan data. Operasi-operasi tersebut berbentuk
baca/tulis ke \textit{file} dan basis data serta permintaan HTTP.

\begin{figure}[!htbp]
	\centering
	\includegraphics[width=1.0\textwidth]{resources/chapter-2/komunikasi
		data pada simulasi.png}
	\caption{Arsitektur komunikasi pada HILS \parencite{trilaksono_laporanRispro}}
	\label{chapter-2-old-hils}
\end{figure}

Dampaknya dapat dilihat ketika dibandingkan dengan SILS (\textit{software
	in the loop simulation}, tanpa NVIDIA Pegasus) didapatkan kinerja 4000 transaksi
per detik. Sedangkan ketika menjadi HILS, hanya didapatkan 100--110 transaksi per
detik \parencite{trilaksono_laporanRispro}. Oleh karena itu, pada subbab ini
akan dibahas beberapa metode komunikasi alternatif yang dapat digunakan untuk
menyederhanakan arsitektur dan lebih cocok dengan kasus penggunaan sistem
simulasi. Alternatif-alternatif tersebut adalah ZeroMQ dan ROS. Kedua alternatif
dipilih karena CARLA mengirimkan sensor secara asinkron dan keduanya dinilai
lebih baik dalam menangani komunikasi asinkron daripada HTTP.

\subsection{ZeroMQ}\label{chapter-2-section-zeromq}

ZeroMQ adalah sebuah pustaka yang memberikan peningkatan di atas
\textit{socket} tradisional. ZeroMQ memberikan fungsionalitas tambahan terhadap
\textit{socket}. ZeroMQ berbeda dengan TCP, tetapi mirip dengan UDP. ZeroMQ
menggunakan pesan untuk berkomunikasi, seperti UDP, dan bukan aliran
\textit{byte} seperti pada TCP \parencite{zmq_chapter2}.

ZeroMQ memungkinkan komunikasi \textit{one-to-one} (seperti \textit{socket}
biasa), \textit{many-to-one} (\textit{socket} mendengarkan ke beberapa
\textit{port}), dan \textit{one-to-many}. Pola komunikasi yang didukung oleh
ZeroMQ adalah \textit{request-reply}, \textit{publish-subscribe}, dan
\textit{pipeline} (\textit{push}-\textit{pull}) \parencite{zmq_chapter2}. ZeroMQ
juga memiliki sebuah antrian pesan (\textit{message queue}) sehingga pesan yang
dikirimkan ke konsumen akan disimpan sampai konsumen siap menerima pesan.
Meskipun terdapat antrian pesan, ZeroMQ tidak harus memiliki \textit{message
	broker} \parencite{zmq_getStartedDocs}\footnote{\textit{zero} pada ZeroMQ
	artinya nol \textit{broker}, nol latensi, nol biaya, dan nol administrasi
	\parencite{zmq_getStartedDocs}.}.

Protokol komunikasi yang digunakan ZeroMQ adalah ZMTP (ZeroMQ \textit{message
	transport protocol}). ZMTP dapat menggunakan berbagai lapisan transpor,
seperti TCP. Semantic sebuah \textit{socket} ZMTP adalah sebagai
berikut \parencite{hurton_zmtp}.

\begin{enumerate}
	\item Semua \textit{socket} akan menerima koneksi dan membuat koneksi.
	\item Semua \textit{socket} akan terhubung ke sebuah \textit{endpoint}
	      secara asinkron dan, jika koneksi terputus, \textit{socket} dapat
	      menghubungkan kembali setelah beberapa saat.
	\item Sebuah pesan harus diterima dan dikirim secara \textit{atomically}:
	      semua \textit{frame} pesan harus dikirim atau tidak sama sekali. Penerima
	      harus mengantrikan semua \textit{frame} pesan yang diterima pada memori
	      sampai semua \textit{frame} diterima.
	\item Sebuah pesan tidak boleh dikirim lebih dari satu kali ke sebuah
	      penerima.
	\item Semua pesan dari beberapa \textit{peer} ZeroMQ di jaringan harus
	      dikirim secara berutuan.
\end{enumerate}

\subsection{ROS 2}\label{chapter-2-section-ros2}

ROS 2 adalah peningkatan dari ROS. ROS adalah kependekan dari \textit{robot
	operating system} dan merupakan sekumpulan pustaka dan kakas perangkat lunak
untuk membangun aplikasi robotik. ROS 2 dibuat karena ROS menunjukkan masalah di
sektor keamanan, reliabilitas, dan penggunaan pada aplikasi skala besar
\parencite{doi:10.1126/scirobotics.abm6074_ros}. Misalnya, ROS memiliki satu
titik kegagalan, sedangkan masalah ini tidak ada di ROS 2.

ROS dilengkapi dengan berbagai macam kakas dan perangkat lunak, mulai dari
\textit{driver} sampai kakas pemrograman. ROS dapat menyediakan fitur serupa
dengan sistem operasi, misalnya abstraksi perangkat keras, kendali perangkat
\textit{low-level}, implementasi fungsionalitas yang sering digunakan,
\textit{message-passing} antar-proses, dan \textit{package management}
\parencite{x_rosIntro}.  Meskipun demikian ROS bukan sistem operasi ``nyata''
seperti Linux, Windows, ataupun MacOS karena ROS harus diinstal di atas sistem
operasi ``nyata'' tersebut.

Setiap perangkat lunak pada ROS diorganisir menjadi \textit{packages}. Setiap
\textit{package} dapat mengandung proses (\textit{node}), himpunan data,
pustaka, dan lain-lain. \textit{Package} adalah struktur terkecil dari sebuah
artifak pada ROS. Sebuah \textit{package} yang mengandung proses dapat
di-\textit{run} untuk menjalankan sebuah proses komputasi
\parencite{x_rosConcepts}.

Proses-proses yang melakukan komputasi pada ROS disebut \textit{node} di level
graf komputasi. Disebut graf karena setiap \textit{node} di ROS saling terhubung
satu sama lain membentuk sebuah jaringan \textit{peer-to-peer}. Sebuah
\textit{node} dibuat agar skalanya \textit{fine-grained} dan bertanggung jawab
hanya pada satu hal. Dalam kasus ini, ROS mirip dengan \textit{service}/layanan
pada arsitektur \textit{microservice} \parencite{x_rosConcepts}.

Mekanisme ROS2 berjalan di atas DDS (\textit{data distribution service}) untuk
komunikasinya. DDS secara bawaan mendukung komunikasi dengan pola
\textit{publisher}-\textit{subscriber} dengan QoS (\textit{quality of service}).
Komunikasi dengan \textit{topic} memanfaatkan pola tersebut. Lalu, untuk pola
\textit{service}, digunakan perpanjangan dari DDS, yaitu DDS-RPC. DDS sendiri
sebenarnya hanya standar dari OMG (\textit{object management group}) dengan
implementasi yang berbeda-beda \parencite{doi:10.1126/scirobotics.abm6074_ros}.
Secara bawaan, ROS 2 menggunakan Fast DDS dari eProsima
\parencite{x_rosRollingDifferentMiddlewareVendors}.

\section{Penelitian Terkait -- \textit{Hardware-in-the-Loop Autonomous Driving Simulation Without
	  Real-Time Constraints}}\label{chapter-2-section-related-works}

Pada penelitian ini \parencite{brogle_CarlaHILS}, dilakukan HILS menggunakan
NVIDIA Jetson TX1 sebagai perangkat keras untuk komputasi dan simulasi
menggunakan simulator CARLA. Salah satu hasil dari penelitian ini adalah
integrasi antara simulator CARLA dengan kakas ROS.

Arsitektur untuk menggunakan ROS dalam simulasi HILS dapat dilihat pada Gambar
\ref{chapter-2-carla-jetson-arch}. Terdapat sebuah \textit{node}  yang
bertugas mendapatkan data kamera dan LiDAR dari CARLA. Kemudian, data tersebut
dipublikasikan ke topik yang didengarkan oleh pemroses data LiDAR dan vidio.
\textit{Node} yang terhubung dengan CARLA akan dikirimkan data terkait kendali
kendaraan oleh \textit{node} yang disebut \textit{path planning node}.

\begin{figure}[!htbp]
	\centering
	\includegraphics[width=1.0\textwidth]{resources/chapter-2/carla-jetson-arch.png}
	\caption{Arsitektur sistem simulasi \parencite{brogle_CarlaHILS}}
	\label{chapter-2-carla-jetson-arch}
\end{figure}

Hasil eksperimen pada penelitian ini adalah hasil simulasi menggunakan HILS dan
CARLA tidak berbeda jauh dengan keadaan dunia nyata. Selain itu, didapatkan juga
waktu respons pada sistem simulasi lebih cepat dari dunia nyata. Sehingga, dapat
disimpulkan bahwa simulasi HILS dapat dilakukan dan implementasi dengan
menggunakan ROS sudah memberikan hasil yang baik.
