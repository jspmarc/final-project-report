\chapter{Studi Literatur}

% Bab Studi Literatur digunakan untuk mendeskripsikan kajian literatur yang
% terkait dengan persoalan tugas akhir. Tujuan studi literatur adalah:

% \begin{enumerate}
%     \item menunjukkan kepada pembaca adanya gap seperti pada rumusan masalah
%         yang memang belum terselesaikan,
%     \item memberikan pemahaman yang secukupnya kepada pembaca tentang teori
%         atau pekerjaan terkait yang terkait langsung dengan penyelesaian
%         persoalan, serta
%     \item menyampaikan informasi apa saja yang sudah ditulis/dilaporkan oleh
%         pihak lain (peneliti/Tugas Akhir/Tesis) tentang hasil
%         penelitian/pekerjaan mereka yang sama atau mirip kaitannya dengan
%         persoalan tugas akhir.
% \end{enumerate}
Pada bab ini, Penulis akan menguraikan hasil literatur dalam penyusunan tugas
akhir ini. Subbab pertama membahas simulator CARLA, yaitu perangkat lunak yang
digunakan sebagai alat simulasi. Subbab kedua menjelaskan NVIDIA Pegasus yang
akan digunakan sebagai mesin untuk menjalankan algoritma \textit{decision
      making} di lingkungan simulasi dan \textit{production}. Lalu, pada subbab
ketiga akan dibahas beberapa cara komunikasi yang dapat digunakan pada sistem
terdistribusi. Terakhir, subbab keempat akan membahas penelitian-penelitian
terkait simulasi \textit{autonomous vehicle} menggunakan CARLA.

% TODO: harusnya ada gambar yg gambarin arsitektur hubungan antara server
% simulator dengan server pegasus

\section{Simulator CARLA}
\blindtext

% TODO: Masukin hubungan dengan NVIDIA Pegasus

\section{NVIDIA Pegasus}

NVIDIA Pegasus adalah salah satu produk cetusan NVIDIA Corporation di bawah
lini produk NVIDIA Drive PX. Nama pasar dari NVIDIA Pegasus adalah NVIDIA Drive
PX Pegasus. Lini produk NVIDIA Drive sendiri merupakan platform komputer untuk
memberikan fungsionalitas bantuan mengemudi pada kendaraan bermotor.

Mengutip dari \parencite{oh_2017}, NVIDIA Pegasus adalah komputer yang
mendukung pengemudian \textit{autonomous} secara penuh. Artinya, NVIDIA Pegasus
dapat digunakan untuk membuat sebuah kendaraan bermotor menjadi
\textit{autonomous vehcile} jika sensor dan algoritma yang digunakan tepat.

NVIDIA Pegasus menggunakan 2 GPU dengan arsitektur post-Volta dan 2 SoC NVIDIA
Xavier. Kombinasi CPU dan GPU ini dapat menghasilkan 320 TOPS (\textit{trillion
      operations/second}) untuk komputasi intelegensi buatan. Untuk koneksi I/O,
NVIDIA Pegasus mendukung sampai dengan 16 kamera (6 di antaranya adalah lidar).

% TODO: Jelasin/masukin hubungannya dengan CARLA dan jelasin juga kalo NVIDIA
% Pegasus ini komputer biasa yang bisa menggunakan berbagai OS (buat transisi ke
% rosbridge)

\section{Metode Komunikasi antara Simulator CARLA dan NVIDIA Pegasus}

Pada subbab ini akan dibahas beberapa cara komunikasi yang dapat digunakan
untuk menghubungkan dua atau lebih komputer berbeda pada sebuah sistem
terdistribusi. Cara-cara komunikasi tersebut adalah menggunakan Rosbridge, RPC,
\textit{messaging system}, HTTP, dan Websocket.

\subsection{Rosbridge}

Rosbridge adalah protokol komunikasi yang menambahkan antarmuka pada komputer
dengan sistem operasi ROS. Antarmuka yang ditambahkan membuat komputer dapat
\textit{publish} dan \textit{subscribe} ke topik ROS dalam format JSON. Selain
itu, antarmuka tersebut memungkinkan memanggil \textit{service} ROS di antara
hal-hal lainnya.

Spesifikasi lengkap untuk rosbridge tertuang di proyek \parencite{ros_bridge}.
Pada subbab ini, spesifikasi rosbridge akan dijelaskan secara singkat.

Secara arsitektur, protokol rosbridge menggunakan protokol WebSocket untuk
lapisan transpornya. Protokol WebSocket sendiri berjalan di atas protokol TCP
yang artinya protokol rosbridge menjamin data akan sampai dengan urutan yang
benar.

Protokol rosbridge menggunakan format JSON untuk pesannya. Pesan yang valid
harus mengandung \textit{field} \texttt{"op"}. \textit{Field} tersebut
digunakan untuk menentukan jenis pesan. Pesan juga dapat mengandung
\textit{field} \texttt{"id"} yang dapat digunakan sebagai penanda transaksi
atau keterhubungan antara beberapa pesan. Selain kedua \textit{field} tersebut,
pesan rosbridge juga dapat mengandung \textit{field} lainnya tergantung jenis
\texttt{"op"}.

\begin{lstlisting}[language=JSON, caption=contoh pesan valid pada lapisan transpor rosbridge]
{
    "op": "operation"
}
\end{lstlisting}

Jenis pesan yang dapat dikirim dapat dibagi menjadi 3 kategori.
Kategori-kategori tersebut akan diuraikan sebagai berikut.
\begin{enumerate}
      \item Pesan kompresi atau transformasi dengan kode \texttt{op}:
            \begin{itemize}
                  \item \texttt{fragment} untuk pesan yang terpecah-pecah, atau
                  \item \texttt{png} untuk pesan yang berupa berkas PNG.
            \end{itemize}
      \item Pesan status rosbridge:
            \begin{itemize}
                  \item \texttt{set\_status\_level} untuk mengatur tingkat pelaporan
                        rosbridge, atau
                  \item \texttt{status} untuk pesan status.
            \end{itemize}
      \item Pesan operasi:
            \begin{itemize}
                  \item \texttt{advertise} untuk menandakan pengirim sedang
                        mempublikasikan suatu topik,
                  \item \texttt{unadvertise} untuk menandakan pengirim berhenti
                        mempublikasikan suatu topik,
                  \item \texttt{publish} untuk pesan ROS yang dipublikasikan ke
                        suatu topik,
                  \item \texttt{subscribe} untuk meminta ``berlangganan'' ke suatu
                        topik,
                  \item \texttt{unsubscribe} untuk meminta berhenti ``berlangganan'',
                  \item \texttt{call\_service} untuk memanggil suatu layanan,
                  \item \texttt{advertise\_service} untuk menandakan pengirim sedang
                        mempublikasikan suatu layanan, \item \texttt{unadvertise\_service}
                        untuk menandakan pengirim berhenti mempublikasikan suatu layanan,
                  \item \texttt{service\_request} untuk \textit{request}/permintaan
                        ke suatu layanan, atau \item \texttt{service\_response} untuk
                        hasil/balasan permintaan ke suatu layanan.
            \end{itemize}
\end{enumerate}

Pesan yang dikirim melalui protokol rosbridge dapat di-\textit{encode} dengan 3
format. Format pertama adalah \textit{raw} JSON. Dalam format ini, pesan yang
dikirim akan berupa \textit{string} biasa. Selain itu, pesan yang dikirim juga
dapat dikompresi dengan format CBOR (\textit{concise binary object
      representation}) atau CBOR-\textit{raw}. Pesan dalam format CBOR akan berbentuk
pesan \textit{binary}. Sehingga pesan harus didekompresi oleh penerima untuk
mendapatkan pesan aslinya.

Perbedaan antara CBOR dengan CBOR-\textit{raw} adalah format pesannya. Pada
CBOR-\textit{raw}, pesan dikirim dalam format serialisasi ROS. Format
serialisasi ROS digunakan untuk mengirimkan data antar-ROS \textit{node} dan
pada berkas \texttt{bag}\footnote{Format berkas yang digunakan untuk menyimpan
      data pada mesin ROS.}. Keuntungan menggunakan CBOR-\textit{raw} adalah
meningkatkan kinerja jika ingin \textit{parsing} hanya sebagian pesan, aplikasi
dapat membaca berkas \texttt{bag}, atau \textit{parsing} pesan ingin dilakukan
setelat mungkin atau secara paralel.

\subsection{RPC --- Apache Thrift}
\subsection{\textit{Messaging System} --- ZeroMQ}

\section{Penelitian Terkait}
\blindtext