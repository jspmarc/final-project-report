\begin{abstract}
	In the rising need for safe, inexpensive, easily accessible, and
	sustainable public transport in a densely-populated country like Indonesia,
	tram becomes a stand-out choice for its government. Along with the rise of
	autonomous vehicle technologies, trams can be made even safer by leveraging
	autonomous vehicle technology to make it operate autonomously.
	Unfortunately, in the development of autonomous trams, testing it on the
	real road would be expensive, dangerous, and time-consuming. Because of
	that, a simulation to test the software and hardware that is going to be
	used in the autonomous tram is needed.

	To do that, a communication mechanism to support the autonomous vehicle HILS
	system needs to be created. The research in this paper details the process
	of creating a communication mechanism for such systems. The communication
	mechanism is packaged in a library to increase its reusability and reduce
	its coupling with the simulation's main programs. The communication
	mechanism proposed by this paper boasts an average latency of only 10.99 ms
	which keeps the CARLA simulator running with 5--13 FPS. This communication
	mechanism also supports the use of CARLA virtual sensors so a simulation for
	autonomous vehicles can be done using the CARLA simulator.
\end{abstract}

\begin{IEEEkeywords}
	HILS communication, CARLA for HILS, autonomous vehicle simulation
\end{IEEEkeywords}