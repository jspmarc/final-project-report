\begin{abstract}
	In the rising needs for safe, inexpensive, easily accessible, and
	sustainable public transport in a densely-populated country like Indonesia,
	tram becomes a stand-out choice for its government. Along with the rise of
	autonomous vehicle technologies, trams can be made even cheaper by
	leveraging autonomous vehicle technology to make it operate autonomously.
	Unfortunately, in the development of autonomous tram, testing it in the real
	road would be expensive, dangerous, and time-consuming. Because of that, a
	simulation to test the software and hardware that is going to be used in the
	autonomous tram is needed.

	To do that, a communication mechanism to support the autonomous vehicle HILS
	system needs to be created. The research in this paper details the process
	of creating a communication mechanism for such systems. The communication
	mechanism proposed by this paper boasts an average latency of only 10.99 ms
	which keeps CARLA simulator running with 5--13 FPS. This communication
	mechanism also supports the use of CARLA virtual sensor so a simulation for
	autonomous vehicle can done using CARLA simulator.
\end{abstract}

\begin{IEEEkeywords}
	HILS communication, CARLA for HILS, autonomous vehicle simulation
\end{IEEEkeywords}