\section{Introduction}

A densely populated country, like Indonesia, needs public transportation that is
safe, inexpensive, easily accessible, and sustainable. One mode of public
transportation that can fulfill these four things is the electric tram. To
reduce potential accidents caused by human error, electric trams can take
advantage of artificial intelligence and control algorithms so that the trams do
not need a driver and can operate autonomously. This implies that trams can have
higher service times, reduce accidents, and increase safety
\cite{trilaksono_laporanRispro}.

The challenge in the development of artificial intelligence technology and
autonomous control algorithms is the need for a large number of tests.
Unfortunately doing the tests for the algorithm is expensive and time consuming
when done in real environment. Furthermore, it could be dangerous if the
algorithm is immature. Therefore, to reduce cost and time consumed doing tests,
the tests is done using simulation.

The simulation is ran on a system with a hardware-in-the-loop (HILS) scheme. The
HILS system requires two computers to run properly: a computer to run the
simulator program and generate virtual sensor data (the CARLA simulator) as well
as a program to run simulation scenarios (called ``ScenarioRunner''); a computer
to be used on the tram which contains a prgoram the control algorithm and
generates controls for the vehicle (called ``GRS''). The computer that runs
CARLA and ScenarioRunner is called SILS, while the computer that runs the GRS
program is called AGX/RKB.

Both ScenarioRunner and GRS need to communicate with each other in order to
exchange sensor data and tram control. The communication mechanism must not be
too slow so that it doesn't limit the simulation system's performance. AGX/RKB
and SILS computers are connected in a local area network (LAN) which implies an
almost ideal environment as the latency and interference in the network would be
minimum. Other than performance issue, it is also important that the GRS program
can use the sensor data generated by CARLA and for ScenarioRunner to consume
tram control by GRS.

Currently, a HILS system for the project already exists. Unfortunately, in the
current implementation, the HILS system has bad performance and it also doesn't
support the use of virtual sensor data \cite{trilaksono_laporanRispro}. This
limits the usage of the simulation system so testing is still mostly done in
real environment. This paper introduces a new approach for the HILS system which
uses a library that connects both programs while keeping the impact to the
system's performance to a minimum. The library is also able to transform virtual
sensor data from CARLA so it is usable by GRS.  With this library, both issues
exisitng in the current HILS system is remedied.
% The library also provides abstraction in the data parsing and exchange process
% in order to provide the best developer experience for the user of the library.