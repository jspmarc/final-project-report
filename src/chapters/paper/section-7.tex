\section{Results Analysis and Discussion}

\subsection{Implementation Testing}

The implementation testing is successful as can be seen in
Section~\ref{section-4-implementation-testing}. This means the new HILS system
can use virtual sensor data for simulation.

\subsection{Performance Testing}

The performance improvement is achieved by eliminating unneeded I/O operations.
The previous HILS implementation needs 8 I/O operations just to send a single
data (the ``theoretical'' version only implements 4). The computers' resource
usage is not yet saturated when the testing is done, which further shows that
the improvement gained comes software adjustments.

Although it has to be noted that there is a technical difficulty in implementing
lidar sensors. Even though the lidar sensor is already usable, as proven in
implementation testing, it inhibits the performance of the simulation. Lidar
sensor slows down the performance because to get the whole lidar image CARLA
needs to run a few simulation steps first, unlike camera and GNSS where only one
step is needed. CARLA needs to run a few simulation steps because to get the
whole lidar image the lidar has to do a full rotation. Each simulation step
only rotates the lidar by some degree, therefore a few simulation steps are
needed to achieve the full lidar rotation.