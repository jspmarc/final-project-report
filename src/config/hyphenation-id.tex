%--------------------------------------------------------------------%
%
% Hypenation untuk Bahasa Indonesia
%
% @author Petra Barus
%
%--------------------------------------------------------------------%
%
% Secara otomatis LaTeX dapat langsung memenggal kata dalam dokumen,
% tapi sering kali terdapat kesalahan dalam pemenggalan kata. Untuk
% memperbaiki kesalahan pemenggalan kata tertentu, cara pemenggalan
% kata tersebut dapat ditambahkan pada dokumen ini. Pemenggalan
% dilakukan dengan menambahkan karakter '-' pada suku kata yang
% perlu dipisahkan.
%
% Contoh pemenggalan kata 'analisa' dilakukan dengan 'a-na-li-sa'
%
%--------------------------------------------------------------------%

\hyphenation {
	% A
	%
	a-kan
	a-na-li-sa
	a-pli-ka-si
	% B
	%
	be-be-ra-pa
	ber-ge-rak
	% C
	%
	CAR-LA
	ca-ri
	% D
	%
	da-e-rah
	di-nya-ta-kan
	de-fi-ni-si
	% E
	%
	e-ner-gi
	eks-klu-sif
	% F
	%
	fa-si-li-tas
	% G
	%
	ga-bung-an
	% H
	%
	ha-lang-an
	% I
	% 
	i-nduk
	% J
	%
	% K
	%
	ka-me-ra
	kom-pu-ter
	kua-li-tas
	% L
	%
	% M
	%
	me-ngem-bang-kan
	% N
	%
	% O
	%
	% P
	%
	pe-ning-ka-tan
	% Q
	%
	% R
	%
	% S
	%
	se-de-mi-ki-an
	si-si
	% T
	% 
	tek-no-lo-gi
	% U
	%
	% V
	%
	% W
	%
	% X
	%
	% Y
	% 
	% Z
	%
}
